\documentclass[a4paper,12pt,titlepage]{article}

\begin{titlepage}
  \title{Core - Work, Employment and Society: Theories, Realities and Methods \\ --- \\Essay Question 6}
  \author{Student Number 1121320}
  \date{April 23, 2003}
\end{titlepage}

\pagenumbering{Roman}
\pagestyle{plain} % plain is default
\usepackage{harvard}

\begin{document}
  \setlength{\baselineskip}{2em}
  \maketitle
  
  \section{Introduction}
  Information and communication technology (ICT) is no longer
  a new concept in the world of management. Therefore, one
  might ask if there are reasons for continue seeing
  ICTs as a problem in the workplace.

  When answering this question,
  I believe it is important to look at all the different elements that
  are affected by the introduction of information and
  communication technology in the workplace. This is because the consequences of ICT
  is far greater than what first meets the eye, when initially assessing
  the concept. I will therefore try to answer this question
  by discussing the different aspects of ICT at work that I believe are the
  most important ones.

  Firstly, the essay will discuss one of the most debated areas of ICTs, which
  is whether or not it means upskilling or deskilling of work.
  Some argue that ICT is the light that management
  have been searching for in order to remove repetitive and
  monotonous work, just as job rotation and job enlargement was the
  ``answer'' in the 1950s~\cite[79]{ahdb},
  whereas other observers claim that Harry Braverman's thesis is still
  true in this new era of information and communication technology,
  arguing that technology is used to
  deskill workers~\cite[119]{vz}.

  A discussion will then follow on the degree of surveillance of work
  that has been made possible by the use of ICT,
  and how this have emerged in more forms and disguises than may first seem evident when
  assessing the problem.

  The essay will then discuss some viewpoints of the trade unions
  regarding the introduction and implementation of new technology in the
  workplace, and how they see its impact on their members.

  I will then go on to assess two other changes
  to the nature of work that has emerged as a consequence of new information and
  communication technology, namely call centres and teleworking.
  Also, I will look at the aspect of how ICT may influence women's
  position in the workplace.

  The last sections of this essay will deal with the impact of
  ICT on management itself and how information and communication technology
  has affected the organisational structure and culture of work.

  \section{Upskilling of Work}
  Many people, especially in the world of management, forecasted that
  the introduction of information and communication technology would
  result in an upskilling of the workers and create multi skilled jobs. It was
  argued that new technology in the workplace would lead to
  job enlargement and more brain work, proving the Braverman's
  thesis wrong, that technology would not cause deskilling and 
  segmentation of the work, separating the ``brain from the hand''.
  And this might be true in
  some cases, especially for workers that have to use
  advanced computer software to do their job and are given
  a certain level of discretion on how to conduct the work~\cite[148]{jw}.

  Huczynski~\& Buchanan (2001) are supporters of ICT, claiming that it has a
  positive effect on work. They argue that technology is often blamed
  for unemployment, as for example in~\citeasnoun{cf}, but argue that
  this is not justified. By looking at unemployment in a broader historic
  perspective, they argue that there is not more unemployment now, than there
  was before ICT came into the workplace. Furthermore, they convincingly argue 
  that technology creates new jobs and gives upskilling to existing
  jobs. The crucial factor for the impact of introducing new technology
  is how work is organised in the organisation and not the information
  and communication technology itself~\cite[70-73]{ahdb}.

  Furthermore, supporters of the upskilling theory, argue that the fear of
  job loss due to new technology is only due to inexperience
  of new technology implementation in the workplace and an ignorance
  about the need for technology change in order to cope with the
  contemporary market~\cite[30]{bl}.

  However, critics towards the upskilling thesis,
  argue that even though technology in some
  cases do lead to upskilling, for the most part
  the opposite is evident. Information and communication technology in the workplace consists
  basically of number punching and filling in the cells of spreadsheets, often
  with the input coming from the telephone. Also, work is
  constantly interrupted by telephone calls, resulting in a very stressing
  working environment~\cite[170]{bbt}.

  When the personal computers started to come into the office in the 1970s
  and 1980s, it was forecasted that there would be a
  revolution and that white collar workers would increase and the number of
  blue collar workers would decline as a direct result of this. Technician
  jobs and engineers were all forecasted an ``outstanding''
  future~\cite[42]{ek}.

  In hindsight, there has been a lot of truth to
  this prediction, however even the technical ``elite'', especially software
  developers and system analysts, have experienced
  both monotony and alienation to their work. Computer programmers
  are not as independent or have such a great degree of autonomy as
  once was announced in the 1970s. Decisions about design and
  implementation are made elsewhere, and the computer programmers therefore
  often perform monotonous and repetitive programming tasks.
  Also, software engineers experience a move towards Taylorism, since computer
  software is demanded to be on time, on budget and with as few errors
  as possible. The new technology has also changed the culture
  of work. Software developers are now expected to work both
  longer and harder hours than before, and not necessarily with
  compensation. Many companies expect the employees to work
  extra hours with additional pay in times of high
  workload~\cite[143-154]{brp}.

  It is not only junior developers that experience alienation in
  the workplace. Even experienced, senior developers have to
  do repetitive and monotonous work. This has two primary
  reasons. Firstly, a lot of development is merely bug fixing
  of existing software, giving little autonomy in ways of creativity
  new development of software. Secondly, all developers
  need to write documentation, something which is
  considered a necessity but a real pain by
  all software developers since this is not what they are trained
  to do, but no one else can do it since the developers made the
  program and are therefore best suited for explaining how
  it works and how to use it~\cite[153]{brp}.

  \section{Surveillance at work}
  Another great concern of the introduction of new
  technology in the workplace is the of surveillance of the
  workforce that is made possible by the new technology~\cite[173]{ch}.

  This criticism has been around for a long time, and
  was commented by Harry Braverman already in 1974
  before the great technology conquest reached the shores
  of the workplace~\cite[77]{ahdb}. 

  The concern for management surveillance of the workers in the workplace
  is also expressed by~\citeasnoun{ho} who argues that the development
  in the practises in using technology as means of monitoring the
  employees is worrying. This surveillance includes monitoring sending of emails,
  telephone usage and internet browsing. She argues that employees
  have both the right to a certain degree of privacy at work, and that
  the employer will benefit from giving the employees a certain
  degree of privacy. Privacy
  gives the employee some degree of autonomy, which again will give
  them a dignity and well being at work, knowing that the employer
  trusts them enough to not peak in their electronic mailbox. Also, Oliver
  argues, it may be beneficial for the workers to develop their
  ideas before announcing them to management or the rest of the company.
  This way, the ideas are of better quality when they are to be assessed by 
  the company~\cite[322-352]{ho}.

  A certain degree of privacy at work also allows employees
  to share intimate information
  among each other, which again creates trust among the workers
  and gives an emotional release. The building of trust is
  very important for the building of effective teams. Strong, mutual
  trust is not developed over night and it is therefore important
  that management gives room for the emergence of this among its
  workers~\cite[294]{sr}.

  Another problem in connection with surveillance at work is
  the conflicting issues it has with the Human Rights Act 1998.
  The Human Rights Act 1998 states the
  freedom to express and privacy.~\citeasnoun{ho} 
  argues this could indeed be applied to the issues of monitoring email
  and internet, although it is questionable what weight this alone would
  have in court. Especially since the converse is true; if the
  the employer makes it clear that email and internet usage are
  not tolerable, there is nothing the worker can do. Also, after
  the introduction of the Telecommunications Act Regulations 2002,
  the employer's rights to monitor the workers is hardly limited
  at all, giving the employer the right to use technology
  in any degree the employer finds fit to monitor the workers
  in the interest of securing company performance and
  profitability~\cite[322-352]{ho}.

  Surveillance at work has changed much the last 20 years, and not
  only directly as a result in information and communication technology,
  but also how the workplace is organised. Clerical work is now
  often conducted in new, ``intelligent'', buildings.
  These new ``intelligent'' buildings represents great usage
  of new technology. Heating, air and ventilation are all
  controlled by computer software. A greater degree of bureaucracy
  is introduced with these buildings, since workers can no longer adjust
  simple working conditions, like temperature, themselves. Earlier,
  workers could just open a window if the room was too hot, whereas
  now, the workers have to direct their inquiry of adjusting the room
  temperature through the organisation hierarchy in order to
  get the matter sorted out. This both represents a re-emergence of bureaucracy
  in the workplace and an example of how the new buildings have become
  an instrument of managerial control.
  The building of new ``intelligent'' buildings are also always
  a top down decision, without any consultation or interaction
  with the workers~\cite[180-181]{bbt}.

  These building have also an important new feature that the clerical
  work did not have before, and that is the concept of open plan offices. These
  offices provides the management with a way of constant surveillance of
  the workers in the office~\cite[166]{bbt} and combined with the
  possibilities of total overview of the work each individual
  worker has done in a given day thanks to ICT, one might argue that Braverman
  was right~\cite[157]{pa}; the main function of
  management is to control the workforce~\cite[63]{hb}.

  Another aspect of managerial control related to new technology, is
  that it keeps the workers physically apart to a greater extent than before.
  Before the computers came to the office, workers had to have more direct interaction
  with each other during the day in order to communicate and get their work done, whereas
  now, they can or must communicate through ICT available to them,
  and management can this way monitor their correspondence through
  their computers~\cite[128-138]{jg}.

  \section{Trade Unions}
  The main objectives of management for introducing new technology
  in the workplace is primarily to get higher production quality and
  greater flexibility. The ability to rapidly change
  the line of production in response to market fluctuations and demands
  is seen as vital in the contemporary world market. The flexibility
  towards change in market demand is also one of the explanations
  of the Japanese success in the 70s and 80s. The usage of
  ICTs for controlling the labour process is only secondary~\cite[22-3]{bl}.

  Even so, there is great mistrust amongst workers towards management's
  real objectives about introducing new technology. When assessing
  whether or not the introduction of new technology is good or bad, it is therefore
  crucial to look at the management's intentions for the innovation~\cite[153]{jw}.

  The primary concern of trade unions is the
  employment of its members, so the concern of new technology
  making workers redundant attracts much of the
  unions' attention~\cite[10]{bl}.

  On the other hand, employees tend to accept new technology at the strategic
  level, that is, they accept the need for change in order to
  reflect the market demands. However, the workers want to
  take part in the decision making at the workplace
  level~\cite[131]{vz}.

  This is one of the reasons why the workers would like to
  be a part of the decision making regarding new technology in
  the workplace. Trade unions tend to be suspicious towards
  technological change in the workplace and this can easily be
  explained by the fact that they are not a part of the decision
  making process~\cite[11-12]{bl}.

  Some places, this has changed though, and the trade union
  is included in the decision making process of new technology.  
  In Australia, unions have been granted the right to
  be consulted when new technology is to be implemented. These
  talks include whether or not the new technology should be
  introduced at all, what sort of technology that is to be used,
  and how management should implement the actual technological changes
  in the workplace~\cite[100-111]{dl}.

  However, this element of employee involvement has its limitations.
  Firstly, there is no real government directive for this that forces the employers to
  implement the practise of including unions in the decision
  making. The companies are only urged to do so by the government.
  Therefore, a lot of companies are hesitant to implement this,
  as the management normally are hesitant to grant the unions more power
  as they see this as a threat to their authority. Another problem,
  is that many unions lack the resources to make a significant
  influence on the employer and at some companies there
  may be more than one union representing the workers and the
  employer will use this reason for completely excluding union participation
  in the decision making process~\cite[111-115]{dl}.

  The trade union situation is
  very different in the US, where there is no nationwide union
  operation as we would find in Europe. Unions in the US usually operate
  on either company or plant level, resulting in a significantly
  weaker bargaining position than in nationwide trade union
  negotiations we can find at the other side of the Atlantic.
  Even worse is that US employers strongly resist trade unions
  when new plants are opened. New technology is used as the
  reason for opening new plants instead of developing existing
  factories. This move opens up for total exclusion of trade
  unions at the new workplace. Technology is therefore
  seen as a real threat to the existence of trade
  unions in the USA~\cite[45]{ek}.

  \section{Teleworking}
  Another dimension that information and communication technology introduces is that it
  diminishes the importance of geography. Workers do not necessarily have to
  be in a certain workplace in order to do their work, thanks to ICT. This aspect
  of ICT may bring work to other workers that previously could not do any work
  at all as they could not get to work, e.g. due to physical disabilities~\cite[199]{gr}.

  Therefore, a new term has come with the emergence of ICT;
  the concept of teleworking. That is, that employees work remotely
  from the office, but are still connected to the usual company network
  and other company resources~\cite[91]{ahdb}. 

  Teleworking occurs in three different forms. The first form is working from
  home, which has a number of advantages. Perhaps the most important
  one is that it gives the employee more flexibility in the time of work. This
  is of great help for working mothers who now can sit at home with a
  computer connected to the company intranet and work as if she was in her
  office. Working from home is a part of the government's new employment
  legislation promoting ``flexible working''.
  From the sixth of April 2003, parents with children under
  the age of six, or disabled children under 18 years of age, can apply
  for flexible working. This includes the right to request to work from
  home~\cite{dti}. Another great advantage of teleworking related to
  equal opportunities, is that it gives greater potential for employment
  for disabled persons, that now can be full time workers and conduct their
  work from home where they have all the extra accessibility equipment
  they may need~\cite[40]{mnpb}.
  
  Teleworking can also be nomadic, which means the employees usually
  work outside the office at the customers' place. Ad-hoc teleworking
  is the last form, which implies that the employee normally work at the
  company office, but sometimes takes the work home in order to be more
  flexible towards increased workload and meeting deadlines~\cite[92-93]{ahdb}.

  Teleworking is by far a new concept, as it appeared already in the 1970s.
  However, it is first now that the costs and specifications of
  the technology is good enough for mass deployment. It is now
  the fastest growing trend in work scheduling and in 1989 it
  was predicted that half of the American workforce would do some
  sort of work from home by the year 2000~\cite[540-541]{sr}.

  In Britain, the number of teleworkers was,
  according to Labour Force Survey, 1.5 million in year 2000, which
  accounts for more than five percent of the British workforce.
  The Bargaining Report from 2001, estimated that about six percent
  of the European workforce were some kind of teleworkers~\cite[40]{mnpb}.

  However, teleworking has also its drawbacks. The initial expenses of
  teleworking is higher for the company since it
  normally involves more investment in equipment for
  the individual worker. And also importantly, this equipment
  will not be shared with other employees at the company, isolating
  company resources to one employee.
  Another natural drawback of teleworking,
  is that since the only interaction with the
  rest of the office is communicated through electronic media, a lot 
  of the social interaction and team spirit is lost. Staff may also
  lose the sense of the organisational structure and may lose track
  of overall company goals. Also, customers may be expect contact
  with a ``real'' office, and may react negatively towards a teleworkers
  representing his or her company~\cite[95]{ahdb}.

  From the worker's perspective, teleworking may feel very lonely,
  since the employee loses a lot of the social network in the company
  and the daily interaction with colleagues. Also, it may get difficult
  to distinguish and separate work life and home life, something both
  the company and the worker's family may find worrying~\cite[40]{mnpb}.

  Another question is whether or not a teleworkers may be disadvantaged
  in office politics and if home workers will be less likely to be considered
  for promotion because of their place of work is not in the company office.
  A question about the worker is whether or not he or she can still get their work
  done at home as good and efficiently as they would have done in the office,
  considering all the distraction that may be in a home,
  such as children, kitchen and neighbours~\cite[341]{sr}. 

  \section{Call centres}
  Call centres is another major industry that has come after the
  innovation of information and communication technology. Call centres
  normally fall into two categories. The first is the help desk, where
  customers can call the company and get help for their problems with
  the company's products or services. The second form of call centres, is
  so called ``interactive processing''. This means that the call centre
  provides a service or sell a product over the telephone. All
  transactions are handled over the telephone and no paperwork is therefore
  necessary.

  The employees working at
  the call centres may be argued to be mutliskilled, since they have to deal with a vast
  number of different problems and questions raised by the calling
  customers. However, the level of absenteeism and staff turnover
  is worryingly high because of the intense stress and pace set out by the
  nature of call centres, which someone
  calls~\emph{``repetitive brain injury''}~\cite[96-7]{ahdb}.
  
  \section{Women}
  The developments in ICT is also of a concern for women in the workplace.
  One of the reasons for this, is the areas in which
  technology makes employees redundant. These areas are especially concerned where
  data automation is taking place, women are heavily represented, that
  is, traditional clerical work of low skilled workers.

  This argument this also supported by~\citeasnoun{pa} who argues that
  computerisation in general deskilled clerical work, and especially
  clerical work traditionally conducted by women, namely banking and
  and life insurance firms~\cite[145]{pa}.

  As with technology in general, the impact on women's work
  situation, is dependent on management's objectives for introducing
  information and communication technology.

  ICT may present a threat to women in the workplace, since it
  it comes from a culture heavily consisting of men with a work
  culture that is no near or even compatible with women's
  lives~\cite[156]{jw}. This was especially
  true in the 1970s and early 1980s, where
  software development was primarily done in dimly lit offices,
  with men wearing long beards and living much like ``computer hippies''.

  Another aspect of the future of women and ICT in the workplace, is how
  computer science is taught in universities and in schools.
  Traditionally, computer science have been a male dominated
  science with teaching methods designed to suit a male
  dominated audience~\cite[155]{jw}.
  
  \section{Management}
  Management on the other hand, has surprisingly also reported
  negative effects of the introduction of ICT. Managers complain that
  the upper management uses ICT as an excuse for requesting a substantial
  higher number of reports and figures from their work, arguing
  that this would mean hardly any more work from the individual
  managers since everything is now computerised. However,
  research has shown that managers complain that IT driven
  management gives increased work load and stress~\cite[173]{bbt}.

  Stress as a negative outcome of introduction of new technology
  in the workplace is also supported by~\citeasnoun{jg} who emphasises
  the importance of understanding the changes in technology in the workplace
  and how this leads to increased stress and pace of work~\cite[127]{jg}.
  The pace of work is therefore controlled by management, since they
  decide to implement the technology which sets the pace. This is also 
  described by Braverman in his thesis on how
  management wants to control the whole labour process in order
  to secure company performance~\cite[195]{hb}. 

  Another aspect of ICT in the workplace is that workers
  in general, and perhaps especially managers, are required
  to have a certain degree of computer literacy in order
  to cope with the latest development. This can also be seen
  in schools, where teachers have to acquire a certain degree
  of computer skills in order to cope with the requirement posed by
  the school~\cite[242]{sc}.

  Strangely enough, even though the company requires its workers
  to keep updated and enhance their organisational and business
  skills, such as ICT, they do not receive any increase in pay
  or formal recognition or career advancement in the
  company~\cite[144]{brp}. The working conditions which the employees
  work within are neither improved because of their upskilled
  work. The claim of the workers gaining or updating their computing
  skills are just implied for keeping their job without any chances
  of other benefits from it~\cite[128]{jg}.

  Another concern of management towards ICT's impact on work, is that
  it seems to  absorb lower level management. This is because new technology
  can by far take over the work previously done by the lower level
  management. Middle and upper level management can now, with the
  help of computers, monitor and guide the work of the employees directly,
  without having to use other levels of management. Also, ICT gives
  workers new tools for coordination and organising their work, so
  in many ways normal workers can operate to a greater extent without
  immediate supervision by lower management who then become
  redundant~\cite[128]{jg}.

  This in many respects, is an upskilling of the workers one might argue, giving
  them both job enlargement and greater autonomy to conduct their work.
  However, as critics point out, there is a ceiling for how much
  management will allow of employee control. Management tend to
  restrict the amount of formal approval of workers' autonomy, although
  it may be expanded by the introduction of information and
  communication technology~\cite[152]{brp}.

  Another consequence of ICT in the workplace that may not
  be so easy to predict, is that it has made it harder
  to climb the hierarchical ladder. Since this traditionally
  have been associated with increased employee skills and efforts, it
  is harder to make a breakthrough with this, since information
  and communication technology skills are expected to be developed
  without any company reward or recognition.
  Therefore, workers have to find other
  ways of justify their case for promotion~\cite[133]{jg}.

  The aspect of no promotion proposals may also have a negative
  effect on the workers' motivation~\cite[176]{bbt}. Early excitement
  about the changes made to the workplace because of IT may therefore
  decline because of no further developments in the individual workers'
  job. Continuous development is important for the workers' motivation
  and the lack of this may be an explanation why many
  successful work teams' performance have stagnated after 
  initial success~\cite[301]{sr}.

  \section{Conclusion}
  ICT came to the office already in the 1950s, and by
  the 1980s it had significant impact on the workplace.
  Previously repetitive tasks were now automated
  by the help of computers, so called data automation.
  In many ways, this process of organising work
  may well be compared to Braverman's description
  of work fragmentation and job deskilling, as many critics of
  new technology argue~\cite[24]{jg}.

  Since the 1980s, information and communication technology have been
  implemented in the workplace in numerous ways, which have
  given both positive and negative consequences for both the workers
  and the management.

  There are massive discussions on the impact of 
  information and communication technology
  and whether or not it represents an upskilling or deskilling
  of work. Even so, there seems to be some sort of consensus
  among the different contributors of the discussion,
  stating that ICT can both mean an upskilling and
  a deskilling of work. It all depends on how it
  is implemented in the workplace~\cite[138]{jg}.

  Information and communication represents not only new tools for
  workers to use, but also new ways of organising
  work and a set of new work cultures. 
  Therefore, there are a number of new issues and problems
  that arise, which management previously did not have to deal with.

  Even though information and communication technologies have
  reached a far more mature status than they had when they
  were introduced in the 1970s, the complexities of ICT are massive and diverse
  and have impact on many parts of the labour process and the workplace.
  Therefore, I will argue that ICT still represents challenges in the workplace,
  which require great attention and concern by management in order to create
  a successful company, that both responds to market demands
  and secures a healthy work environment for its employees.

  \newpage
  \setlength{\baselineskip}{1.3em}
  \bibliography{essay}
  \bibliographystyle{jmr}
  \nocite{*}
\end{document}

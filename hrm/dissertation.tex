% $Id: paper.tex,v 1.5 2003/08/17 19:05:03 torstein Exp $

\documentclass[a4paper,12pt,titlepage]{report}

\usepackage{natbib}
\bibpunct[:~]{(}{)}{;}{a}{,}{,}
\pagenumbering{arabic} % Roman
\pagestyle{headings} % plain is default

\makeindex

\begin{titlepage}
  \title{Predicting a Team's Behaviour by Using Belbin's Team Role Self Perception Inventory  \\ --- \\Human Resource Management \\ Dissertation}
  \author{Torstein Krause Johansen \\ Department of Management \& Organisation \\ University of Stirling}
%  \date{August 31, 2003}
\end{titlepage}

\begin{document}
%  \sf % sans serif font
  \setlength{\baselineskip}{2em}
  \maketitle

  \begin{center}
    \bf{Acknowledgements}
  \end{center}
  This dissertation has been written with the help of a number
  of people whom I wish to direct a warm ``thank you''.

  The department of Human Resource Management was naturally my
  primary resource of consultancy and I would therefore
  like to thank Dr Stewart Butts,
  Phil Taylor, Juliette Summers and Dr Simon de Tumberville for valuable
  feedback on my dissertation progress.

  When starting this project I also realised that I needed guidance
  to the world of psychology. Therefore, PhD student Mark Boydell
  introduced me to Dr Fiona MacCall and Dr Ian MacFarlane at the Department
  of Psychology at the University of Stirling. They were all greatly
  helpful in both providing me useful tips to where to look and 
  how to conduct my research from the perspective of social psychology.

  My research was conducted at the Student Association executive board
  and in that connection I would like to direct a special thanks
  to Declan Walsh, Richard Pyle and Helen Batey who were all
  greatly helpful. An acknowledgement should also be given to
  all members of the Stirling University Student Association (SUSA)
  executive board and the SUSA administration.

  \begin{abstract}
    This dissertation assesses the validity of Dr R. Meredith Belbin's
    famous team role self perception inventory (BTRSPI). After giving
    a review of previous research on teamworking and the BTRSPI, it
    attemps to link the BTRSPI profile of a real decision making team
    to the team's behaviour when conducting their normal work in
    their genuine environment.

    It was found that more often than not, 
    individual team members' behaviour is in accordance with their
    BTRSPI profile.

    The study carries relevance to both human resource management and
    the general discussion of the BTRSPI as the BTRSPI is the most widely
    used measurement by management when composing teams and when
    assessing team performance.

    This research contributes to the BTRSPI validity discussion as it links
    the BTRSPI to the behaviour of a real team in their genuine organisational
    setting.
  \end{abstract}

  \tableofcontents

  % chapter 1
  \chapter{Introduction}
  Teamworking is becoming increasingly important these days. According to a UK
  survey, it is one of the essential skills for entering the world of
  management~\cite[61]{sfwmgs}. Therefore, it has become increasingly desirable
  for management to be able to predict team performance, that is, how others
  perceive the team output~\cite[399]{ahdb}, and to know how to
  compose a team that functions well and gives the desired output.

  In this connection, the Belbin team role self perception inventory (BTRSPI)
  has been the centre of much attention. It attempts to give the one taking the test
  a description of what kind of team role he or she has by simply calculating
  the scores of a  self perception questionnaire.

  The BTRSPI has been widely adopted by management because of its simplicity and
  intuitive appeal. Therefore, it has been subject to a massive discussion of its
  validity because of its wide spread usage. Much research have been conducted on
  the BTRSPI; what kind of team roles UK managers tend to have, gender differences
  within the BTRSPI profiles and if there is a link between the BTRSPI and
  what kind of university career people choose.

  More interestingly is the research conducted to link the BTRSPI to the
  actual performance of a team, that is, how well a team functions,
  how fast they find an answer to a given problem, and the quality of that
  answer. However, most of these studies have been conducted on artificial teams,
  that is, teams that are not real teams outside the study. This is bad as it
  takes away the dimension of in-group relationships formed over time and the
  implications of team decisions on their organisation and on the team's inner
  workings.

  The topic this dissertation therefore wants to address is how well
  the BTRSPI scores of a real decision making team corresponds to the
  actual behaviour of the team in their genuine setting.

  The study was carried out on an executive board of the student organisation
  at the University of Stirling. The board members were to take the BTRSPI and were
  observed in their board meetings in addition to being interviewed about their
  previous and current organisational experience. The results from the two first 
  methods were then compared to see if there was a clear link between them,
  or if unexplainable discrepancies occurred to a great extent.

  It was found that the BTRSPI carried a respectable validity. In more
  than half (67\%) of the team members' interactions in the board
  meetings could be directly linked to the characteristics of
  their Belbin team roles. However, there were a significant
  number of incidents (32\%) that were true opposites of the behaviour to be expected
  according to the executives' BTRSPI profiles.

  The findings have relevance to the discussion of the validity of BTRSPI as it
  is the first linkage made (to the author's knowledge)
  between the BTRSPI scores and the behaviour of a
  real team in their real setting. Also, the study should be interesting to human
  resource managers as they often have the organisational responsibility to build
  and assess teamworking~\citep{fzje}.

  \section{How the dissertation is organised}
  This dissertation consists of five chapters in addition to this
  introductory chapter. The next chapter will give a  review of
  the current literature on Belbin, teamworking and in particular;
  decision making in teams. The chapter will give a presentation
  of the different perspectives of previous authors and present
  empirical findings of their research that are relevant to the area of the
  dissertation. These will include the ones that will be questioned or
  challenged in this dissertation. The chapter will then go on to
  draw some pieces of criticism of previous authors' assumptions,
  perspectives and arguments.
  
  The following chapter will present the methodology used in the
  research. It will give a detailed explanation of the research objectives
  and a rationale for the research strategy chosen to tackle the research
  objectives. The chapter will then describe the key concepts of
  the research and how these are defined and measured. It will go on to
  present the theories and hypothesis set before the 
  research was conducted. A detailed outline of the methods used
  and the reasoning behind them will then follow. The chapter will
  be rounded off with a discussion of problems or difficulties
  encountered when planning and conducting this dissertation
  research, together with suggestions of further research.

  Chapter four will present the results from the dissertation research.
  Chapter five will give a discussion of the key findigs and compare them
  with the dissertation research objective and previous research. The
  chapter will then go on to discuss the implications of the findings
  for human resource management and the SUSA executive board.
  The chapter will end off with suggestions on
  how future research may be developed with reference to this one.

  The concluding chapter of the dissertation will give a summary of
  the dissertation topic, the research conducted, the findings
  and conclusions that can be drawn upon these results in relation
  to the BTRSPI in general and the dissertation question itself.

  %chapter 2
  \chapter{Literature Review}
  \section{Introduction}
  % an introduction that clearly indicates the aims of the review
  % and the reasons for the structure you have chosen.

  The aims of this literature review is to give a summary of what have previously
  been done by other authors that have relevance to this dissertation topic.
  This chapter will first give a presentation of the Belbin's team role self
  perception inventory as this theory plays a key part in the dissertation. It will
  then go on to discuss both the different theoretical perspectives of
  previous authors and also present the main empirical findings of their research. 

  The text will then explain how this research fits in with the greater perspective by
  pointing out criticisms and shortcomings of previous authors' work and how
  this dissertation research can contribute to the topic. The chapter will then present
  the team used for the research and conclude with the reasons for
  how the research itself was structured.

  \section{Key theoretical and empirical issues}
  The key issues involved naturally include Dr R. Meredith Belbin's theories
  on the composition of teams. His model of how to construct a team that
  has the right balance of different skills needed is one of the empirical
  cornerstones of this dissertation.

  In that connection the BTRSPI is
  of uttermost importance as it is the physical connection to his theories in most
  research done to test the validity of his team role theory.

  Other key theoretical issues involved includes groupthink. Teams tend
  to make decisions that are not optimal when considering the amount of knowledge
  and experience the teams holds. However, due to the human nature and
  the exposure to internal group processes, irrational decisions do
  occur \citep{mhsh,rb}.

  The research also deals with social pressure to conform to the group norm.
  People in general do not want to be unpopular, and  deviants are liked less
  than the other team members and is therefore desirable to avoid.
  Also, deviates are exposed to a greater pressure
  from the rest of the group. A person with unpopular views is addressed far
  more often than the ones with the general viewpoint. In order to avoid this,
  people tend to conform to the group norm~\cite[141-142]{rb}

  \section{Why the topic is important and to whom} 
  Teamworking with discretion for decision making is becoming increasingly important
  in the workplace. Therefore it is of great interest for management,
  to be able to predict how a team will function as decision makers.

  The BTRSPI was included in \citet{rmb} as an appendix. It is a test
  that finds the team role of the one taking the test,
  as described in \citet{rmb} and \citet{rmb2}.
  This inventory is widely used by management, to measure
  team performance, to train teams, to compose effective teams
  and to track problems in teamworking in the company~\cite[41]{af}.

  Since the BTRSPI is so much used by management, it carries a responsibility
  for being accurate~\cite[261]{afhsdp}. Also, it is important to
  find in what kind of ways it is applicable and in what areas it is not.
  This dissertation contributes to the discussion of the
  applicability of the BTRSPI in a setting of a decision making team.
  The topic is important to all levels of management that are in some way
  connected with teams. This is especially evident in departments
  responsible for composing and training teams.
  Often this is the company's human resource management
  department~\citep{fzje}.

  \section{Belbin's team role self perception inventory\label{btrspi}}
  In 1981, Dr R. Meredith Belbin published his
  book ``Management Teams - Why They Succeed or Fail''.
  He identifies eight different team roles that have to be present in order
  to make a balanced and effective team. The team does not necessarily have to consist
  of at least eight members, on the contrary Belbin strongly argues that the
  groups of six is the sacred number for small groups~\cite[115-7]{rmb}.
  Nevertheless, the different
  roles have to be present in the team members. A team member can
  hold several roles, contributing to a balanced team. Each
  of the roles has its strengths and ``allowable'' weaknesses.
  Belbin also describes what other team roles the individual roles function well
  together with and what combinations of roles may pose conflicts~\citep{rmb}.

  The ``chairman'' (CH) is the first of Belbin's team roles. Persons fitting this role
  tend to be calm, self confident and controlled leaders. They are to guide
  and control the other team members in the teamwork situation.
  Chairmen have a strong sense of objectiveness and thus
  welcome all potential contributors in the team.

  While the ``chairman'' does not have any ``allowable weaknesses'', as Belbin labels
  them, the ``shaper'' (SH) is impatient, provocative and may be irritated during the
  team meetings. They tend to be highly stung and are demanding, coercing and
  confrontational leaders that wants to push the others in the team to excel. The
  shaper has a strong drive, is dynamic and has the courage to overcome obstacles.

  A close relative to the ``shaper'' is the ``plant'' (PL) who is the typical innovator
  and problem solver.
  This is the team member that is typically the new idea inventor. Plants are
  individualistic, serious and unorthodox. Their genius, imagination, intellect and
  knowledge are their strong sides, whereas their ``allowable weaknesses'' are that
  they ignore the practical implications of the team's problems and have a strong
  personal relation to their ideas.

  The ``resource investigator'' (RI) is the team's contact person for external sources
  of information. Resource investigators have great communication
  skills, are enthusiastic,
  extrovert and are eager to explore new alternatives and respond to new challenges.
  However, they are liable to lose interest after the initial fascination has passed.

  The fifth team role is the ``monitor-evaluator'' (ME) and is the person who analyses
  and evaluates the proposed solutions and choices in the team. They tend to be sober,
  prudent without any emotions related to the teamwork tasks.
  Therefore, monitor-evaluators
  have clear judgement, discretion and can be seen both as cynical and sceptic.
  The negative aspect or the monitor-evaluator role is that they lack
  inspiration and do not have the ability to motivate others.

  The ``company worker'' (CW) is the person in the team that implements the plans
  agreed in the team meetings. The person is conservative, and can be counted on to 
  carry out all his/her duties. The company-worker has great organising skills, is
  hard working with great self discipline and has good
  portion of practical common sense.
  On the other hand, the person does not have flexibility or responsitiveness to
  unproven ideas.

  The least surprising Belbin team role is the ``team worker'' (TW).
  The team worker facilitates the core team functions and is the mediator within the
  team. This person is socially skilled, mild and sensitive to the other team members'
  feelings and is able to respond to whatever the team is presenting and promotes
  team spirit. However, the person lacks resolve in times of crisis.

  The eighth and last of Belbin's original team roles is the ``completer-finisher'' (CF)
  who keeps focus on details and deadlines. Therefore, the person is typically
  painstaking, orderly, conscientious and anxious. As the role name implies, the
  person has a great capacity for following it all through and keeping attention
  to detail all the way to the very end of the project. As a natural
  consequence of this, or labelled as a ``natural weakness'' by
  Belbin, the completer-finisher has a
  tendency to worry about minor details and is
  reluctant to ``let go''~\cite[243]{shts}.

  A ninth role, the ``specialist'', was added by Belbin in 1993.
  This role is described as a single-minded person that
  is dedicated to his area of expertise and the topics
  discussed in the team related to this area. The person gives the team
  valuable and rare skills and knowledge. However, the person may
  dwell too long on technicalities and the overall contribution to the team
  is narrow~\citep{rmb2}.

  As an appendix in \citet{rmb}, an inventory was supplied, the BTRSPI,
  which could be used as a guidance for finding these team roles
  described in the book. Due to copyright, the inventory could not
  be included in the dissertation.

  % tkj: does this fit in here?
  The BTRSPI is one, if not~\emph{the} most used measure of team role 
  preferences in the UK~\cite[33]{sbab}. Its wide usage is both due to
  its simplicity of use; the participants can do the test without
  any special arrangements or observation, and its intuitive
  appeal to the one interpreting the results~\cite[66]{sfwmgs}.

  However, the inventory was never meant to be a
  full psychometric test on its own, as pointed out by Belbin in the debate of the
  validity of BTRSPI~\cite[259-260]{rmb3}. Nevertheless, its
  wide spread usage by management is unquestionable~\cite[33]{sbab} and proof has
  been produced by several authors, e.g. \citet{sfthwm3},
  \citet{sfwmjw}, \citet{sstm}, \citet{bwmg}, \citet{ysht} and
  \citet{jssd}, to confirm its validity.

  \section{Theoretical perspectives of previous authors}
  % a discussion of the theoretical perspectives of previous authors in
  % your field or related fields.

  This section will give a summary of the main areas of
  research related to the BTRSPI and decision making in teams. It will first
  present some of the different perspectives held by researchers on teamworking
  in organisations, as this was the starting point for this dissertation.
  The text will then describe some of the work done
  on gender and teamworking. It will then go on to summarise
  what has been done with linking Belbin to
  teamworking in organisations before presenting the major theoretical perspectives
  of previous authors in assessing the validity of the BTRSPI,
  which is where this dissertation research fits in.

  \subsection{Teamworking in the organisation}
  Teamworking has been subject for a tremendous amount of research and is still
  a popular and yet not thoroughly explored area of management studies.
  \citet{escbjc} assessed what teamworking really is and reviewed the
  different efforts made in the management literature to defined what it is.
  Among other things, they found that decision making is one of the core
  skills in teamworking~\cite[343]{escbjc}. 

  Another element that several authors conclude related to teamworking, is
  the importance of clearly defined team roles, as found in 
  \citet{rdjlno} and \citet{cg}.
  
  Another skill that is found by researchers to be of the highest importance, is
  human relation skills, as described by \citet{dphh}. The authors researched the
  implementation of teamworking in the car industry. In addition to measure the success
  of implementing teamworking in the industry, they found evidence for the need of
  teamwork training. They emphasise that getting successful teamworking
  is not as easy as just putting people together and expecting them to function
  as teams.

  One popular perspective of authors, e.g. \citet{pfamampt},
  is to look at a company or an industry at large, and assess the
  impacts of teamworking on the the environment as a whole. \citet{nbpb}
  looked at the iron and steel industry and investigated
  how teamworking had change the way work was conducted. They concluded that
  workers' positive experience and also company performance increase were dependent
  on management's objective for introducing teamworking. 

  \citet{lg} researched the success of implementing teamworking in Australian
  industry. She found that the introduction of teamworking in the workplace can be
  successful~\footnote{In this connection, success is measured at two levels; company level (increased labour productivity and a flatter management structure) and employee level (decreased labour turnover) A more generic definition of the success of a team is how others
  perceive the team output~\cite[399]{ahdb}.},
  but that it depends on the type of work. Not all jobs are compatible with teamworking.
  Indeed, as with any implementation of work and organisation design,
  teamworking is more appropriate to some settings than
  others~\cite[1412]{fmspdb}. Other writers argue that
  ``it is entirely possible to force a team-based form of work design on to a
  process with non compatible characteristic'', however it is concluded that
  work designs should reflect the feature of the production
  process~\cite[1538]{cspjsp}.

  \citet{cspjsp} did similar
  research on employees in wire mills and looked at the success of teamwork
  implementation. \citet{wrc} supports this group of researchers claiming
  that teamworking is not a fast fix and that organisations must give teamworking
  a chance in relation to time and resources.

  The success of teamworking in the workplace is also explored by \citet{mcsarc}
  who interviewed members of cross functional teams in large multi site companies
  on the progress of this new element (teamworking) of the work organisation.
  The focus of these authors was the learning outcomes
  of both the organisation and the employees themselves.

  \citet{kg} researched the positive impact of implementing nursing teams.
  It was found that the communication became more effective, and it was
  concluded that teamwork can provide better primary care.
  In this connection, decision making was one of the successes found of
  implementing teamworking.

  The increasing usage of teamworking in the health service is further confirmed
  by \citet{wh} who states that many hospitals are modifying their whole
  organisational structure to embrace teams.

  Research on the effectiveness of management teams was
  conducted by \citet{rjsbjb} where the aim was to find a positive link
  between extraversion and conscientiousness to self-efficacy for participating
  in teams and attraction to the team.

  Another perspective on team composition is taken by \citet{jcscjb}
  who looked at the composition of school leadership teams, where they had different
  members of the groups, including students and parents in addition to school staff.
  The study also studied the impact of team member training in group processes.

  \citet{jhdi} researched the importance of feedback to a team for ensuring
  team performance. A similar study was conducted on a university hospital by
  \citet{khka}. They found that supervision improved the decision making in
  the teams.

  \citet{dkdm} wanted to explore the impact of teamworking for employees in
  the automobile industry. The authors argue that because there is no single form of
  teamworking, there is no single experience of what teamworking means to an employee.
  One interesting finding here, was that some employees,
  whilst being committed to teamworking, were actually aware of
  that it required from them a psychological change.

  \citet{nlrskk} tried to find the correlation between the four most used
  schemes for measuring the cognitive styles of decision making. They found that
  even though there is a lot of overlapping in their theory, there were few strong,
  inter-relationships between them.
  
  \subsection{Gender and teamworking}
  Another perspective of research on teamworking is the element of gender
  differences in teams. \citet{jljhdijcae} looked at how the gender composition
  of a group influenced the decision making in the team. Performing traditionally
  masculine tasks, they found that the team decisions grew more and more
  aggressive as the percentage of male team members increased.

  The gender composition is also noted by Belbin himself, who argues
  that men and women and different problem solving preferences, which again
  will influence the team decision making process~\cite[41]{cg2}.

  % link from gender to org. section.
  \citet{jssd} explored the linkage between team role preferences and the kind
  of study selected by students and if there is a difference between male and female
  students. The majority of males were implementers (25\%) or co-ordinators (23\%),
  whereas the majority of women were team workers (45\%). Also, the majority of
  business and occupational therapy students were team workers (24\% and 50\%
  respectively)~\cite[170]{jssd}.

  \subsection{Belbin in the organisation}
  More interestingly, in connection with this dissertation,
  is the linkage to Belbin's team roles and decision making in teams,
  such as in \citet{sbab} who did research on both public and private sector
  organisations. They searched for a pattern in team role preferences and the choice
  of occupations. 

  The Belbin team roles have been applied to UK managers and \citet{sfthwm2}
  found that the distribution of team roles in UK managers were mostly
  co-ordinators and resource investigators, whereas there were few
  completers, monitor evaluators, plants and shapers.
  
  The same authors also wanted to prove that Belbin's team role theory also
  could be applied to non managers also, as much of previous research have
  greatly focused on Belbin's team roles in a managerial setting. The participants
  were from different public organisations, such as councils and the
  National Health Service (NHS). They argued
  that the decision making process is equal at higher as well as lower levels
  in the organisation~\cite[20]{sfthwm3}.

  Other perspectives of Belbin can be found in the research
  conducted by \citet{sfwm}. They wanted to find out if there
  was a link between a person's upbringing and the team role preference later
  in life. The authors did find a link from a person's family
  environment during childhood and the preferred team role later in life.
  However, as the authors point out, the research had two drawbacks;
  the data collected was from self perception reports
  and two; the participants were students and not real managers.

  \citet{shts} used software engineering students to conclude that teams with
  Belbin's leader roles performed better than the ones without these roles present.

  \citet{ysht} wanted to prove that managers' team role preferences 
  were dependent on the task environment in which they worked. They argued that
  an organisation's culture and nature will influence the values and preferences
  of its managers. The authors found a clear link between team role preferences and
  the different environmental dimensions defined in an organisation.

  \subsection{The validity of Belbin}
  A lot of the research related to Belbin, is to test its validity, and it
  is here this dissertation has its primary focus.

  One of the reasons for the massive amount of research and criticism on the validity
  of the BTRSPI is because of its wide usage by management in the industry. Due to
  its extensive usage, it carries a responsibility of being a valid
  model~\cite[261]{afhsdp}.

  There has been a number of research conducted on the validity of
  Belbin that suggests that the inventory is invalid.
  \citet{cj} compared Belbin's BTRSPI with Honey and Mumford's learning
  styles questionnaire. The question asked was which one best predicted the
  team's performance. However, in this research, the writer did not find
  the BTRSPI predictive.
  %  a more valid criticism of the results of this study is that the management
  %  teams were created specifically for this study, whereas real-world management
  %  teams may operate for months.

  \citet{wmdh} studied the darker side of team roles, and found that the
  the ``allowable weaknesses'' that Belbin describes regarding the different
  team roles may not be so ``allowable'' after all as they can damage
  the health of the team itself.

  According to~\citet{dphh}, one
  of the few pieces of research that has used real teams as research participants
  has been conducted by \citet{bsss}.
  \citet{bsss} compared the results from the BTRSPI and
  another tool of Belbin, the ``Observer's Assessment Sheet'', on members of
  11 different management teams. They found positive correlations between the
  two tests for only five of the nine team roles. The authors warn
  that one should not emphasise self perception tests too much
  when selecting and employing people~\cite[1-7]{bsss}. 

  A lot of research have found evidence for validity of the BTRSPI. Evidence
  has been presented that suggests that the Belbin team role model has validity when
  it comes to predicting more powerful and involved team members~\cite[586]{sfwmjs}.
  \citet{sfwmgs} also wanted to add weight to the validity of the BTRSPI,
  by comparing the scores of Cattel's 16PF personality questionnaire. They
  found evidence of the usefulness of the BTRSPI, although 
  the importance of using other measures in addition to the BTRSPI
  when predicting team performance was emphasised~\cite[67]{sfwmgs}.
  This conclusion was also reached by \citet{sfwmjw}.

  \citet{dphh} found support for Belbin when it comes to the presence or
  absence of some individual roles. The balance of the team did give a negative
  or positive effect on performance. However, the study also found that the duplication
  of roles is not necessarily bad in itself, it depends on what kind of
  role it is~\cite[703]{dphh}. This argument is supported by \citet{jpns} who explored
  the performance difference between teams with a balanced composition of team roles,
  and teams with several or no shapers. Using the shaper role as the deviant was not
  a coincidence as the shaper is described as the most disruptive person
  in the group. The authors found that teams with a balance of roles performed, as
  claimed by Belbin, better than unballanced teams~\cite[654]{jpns}.

  \citet{sstm} reinterpreted a sample of 5003 scores of the BTRSPI and
  found that the internal consistency was much better than previous research
  have suggested~\cite[529-533]{sstm}.

  It is also found that BTRSPI can be useful when identifying
  strengths and weaknesses in a team that is to solve a task, allowing
  tasks to be allocated based on competence~\cite[105]{bwmg}.
  The authors observed a group of seven undergraduate
  consultancy project team members studying in their final year of business,
  working with a real client and a genuine business problem.

  
  \section{Summary of main empirical findings of previous research}
  Naturally, \citet{rmb} and \citet{rmb2} are very important to
  this dissertation research as they represent the framework for the
  dissertation topic.

  The research by \citet{bsss} is in some ways the closest thing the dissertation's
  research. However, the study is not relevant to mine
  in that it only tries to link the scores of Belbin's
  ``Observer's Assessment Sheet'' and Belbin's team role self perception
  inventory and not to predict or explain the actual behaviour of the
  team.

  A very interesting finding is that Belbin's ``allowable weaknesses''
  of the different team roles may in fact not be ``allowable'' as they
  can damage the core health of the team spirit, which again can affect
  the decision making~\cite[362]{wmdh}.

  The research of \citet{sfwmjs} is very interesting and important
  for this dissertation as it found good evidence for the validity of
  BTRSPI. The research concluded that the BTRSPI
  is useful for predicting powerful and involved team members.

  In this connection, research that questions the validity of Belbin, such
  as \citet{cj}, are of importance to this dissertation topic.

  Another main empirical finding is that of \citet{dphh} who found
  differences in team performance when some of the team roles were
  absent. The study also found that the duplication of roles is not necessarily
  negative as it depends on the role in question.

  Thus, the study answered a question by posing another question. Therefore
  the findings of \citet{jpns} are very interesting as they suggest
  that an imbalance of shapers can affect the team to a great extent. This
  argument is one of the elements this dissertation assesses when trying to link
  the BTRSPI scores with the observation of the SUSA exec board.
  

  \section{Criticisms of other writers' assumptions, perspectives and arguments}
  There has been a lot of research related to the BTRSPI.
  However, the research encountered in my readings have had one of the
  following weaknesses described below.

  The first of this dissertation's main criticisms is that a lot of the research
  conducted have been solely based on the self perception inventory without anything
  else to back it up. Other writers have chosen other inventories or schemes to
  test the validity of Belbin. However, they still do not have any real life
  observation to compare them with.

  Some writers again have an observation of the participants that took the
  BTRSPI, however this is conducted by either the team members' peers or
  more than one independent observer.
  The drawback of this, is that what one person perceives
  as a certain kind of behaviour may not correspond to another's.

  % tkj chop up the paragraph into sentences. Check again.
  Another criticism, is that the research itself has either been
  conducted on real life teams where there has been no observation of their
  actual behaviour in an authentic team situation, e.g. \citet{kg}.
  Some research do address this problem, but does so by putting the
  team in an unfamiliar setting and by doing so removes some of the authenticity of
  the research.

  Other research do have the team in their normal organisational environment, but
  compose the team out of people that do not normally work in a team. The drawback
  of this is that decisions in the team will not have any effect outside
  the research arena and the team members may behave differently compared
  to a real team as a consequence of this. 

  The last and most significant criticism is that in research were
  the authors do link the BTRSPI with observation of a team's behaviour,
  the research participants are not real decision makers from an organisation.
  It seems that the use of university undergraduate students is
  the primary source of research participants
  in the field of group processes and decision making in teams. This drawback is
  sometimes mentioned by the authors themselves, e.g. \citet[14]{sfwm}
  and \citet[419]{nlrskk}.

  The honourable exception to this criticism is \citet{bsss}. However,
  in this research the observation was done by peer review and not by an
  objective third part. Personal emotions or other organisational
  elements such as the positions the members hold
  could therefore get in the way of an objective observation.

  \section{SUSA}
  The team that was the focus of the dissertation research was the
  Stirling University Student Association (SUSA).
  The board consists of 11 members, four of them
  are full time, paid employees. These four members of the
  board have votes at the different university boards and committees.

  The cases discussed on the SUSA executives board meetings
  have direct impact on how the whole SUSA organisation is run,
  including student accommodation, child care, medical services,
  campus events, the student bar, the student disco and
  miscellaneous campaigns on campus.
  Also, the board meetings give the different representatives that are responsible
  for the different parts of the SUSA organisation the opinions to
  bring forward to the university committee and council meetings, including
  the university academic council and the university court.

  The great advantage of doing the research on the SUSA executives board
  was that this is a real decision making team. Each executive board
  operates for at least a year and the decisions made in the meetings
  have real impact on the SUSA organisation and on all students at the university.

  \section{Concluding summary}
  %  concluding summary of your reasons for the structure and direction of your
  %  own research.

  As far as what has been covered in this literature review,
  there has been no attempt to link the scores from BTRSPI with observation of
  the very team of participants in their natural environment conducting their
  normal work as decision makers. Therefore, this is what this
  dissertation will address.

  The dissertation research started with getting a Belbin profile of the team.
  The executive board had therefore to take the BTRSPI.
  The team was then observed under their weekly meetings over a two month
  period to see if there was any correlation between the BTRSPI scores
  and the team's behaviour. The BTRSPI test was conducted anonymously
  so that the persons behind the different scores were unknown during the
  observation.
  Also, interviews were made with all team members to get a experience and
  general background of each of them, in order to get a greater
  understanding of how the team worked. 

  By using the combination of the elements mentioned above, this 
  dissertation tries to give a new flavour to the research done on the validity
  of the BTRSPI as it tries to see if there is correlation between
  the BTRSPI scores and the actual behaviour of a team when doing their
  genuine job in their genuine organisational context. 

  Also, it was interesting to see how the role of the deviant was played, if it was
  the same persons that were deviants and if it was possible to see a forming of
  smaller groups within the team. As noted e.g. in \citet{mhsh}, team members
  that share friendship outside the team setting may follow the same pattern inside
  the group setting. Therefore, another objective of the research was to see
  if there were any ``allies'' in the team that could be predicted from the
  Belbin team roles as he described roles that worked well together and roles that
  did not.

  The research therefore wanted to answer the question:
  \begin{quote}
    \bf
    Is it possible to predict the behaviour of a decision making team
    from the team role profile given by Belbin's team role self perception test?
    \sf
  \end{quote}
  
  %chapter 3
  \chapter{Research design and methodology}
  \section{Research strategy}
  The research strategy chosen for this dissertation was selected from
  the perspective that it is necessary to get in-depth knowledge of
  the team. A thorough understanding of both the team members and
  the organisation which they operate within are
  fundamental in order to meet the research objective; 
  to see if the team's BTRSPI profile can predict the behaviour of the board
  members in their meetings.

  For this reason, the dissertation research is of a qualitative nature. 
  A qualitative approach makes it possible to gain a thorough understanding
  of the group processes and team role profiles of the executive board.
  Furthermore, it makes the research more manageable in terms of
  coordination of interviews, questionnaires and observing the
  team meetings. When observing team meetings it is important to
  have the right meta data about the team fresh in mind. Hence, what the
  members have said in their interviews and the team's BTRSPI profile must
  all be clear to the observer before entering the team meetings for
  conducting the observation.

  Under other circumstances, it could have been desirable to observe
  multiple teams, in order to get a broader
  fondation to draw dissertation conclusions on. This would solve
  one of the drawbacks of basing research
  on only one group. This is that the group may not be representable for
  the kind of group it is. Exceptions do occur, and it may happen that
  this one team is the exception to the rule. However, due to the 
  the complexity of coordinating interviews, BTRSPI questionnaires and observing
  multiple teams without confusing the scores and observation transcripts
  between different teams, it was decided to base the research on one team
  only.

  As outlined in chapter 2, this research wants to address the
  gap in previous research done on BTRSPI and team behaviour.
  To accomplish this task, the team needed to be a real one and not
  just one composed for the occasion, which has been the norm
  in previous research on this topic.

  Furthermore, the team
  had to be observed in their genuine organisational environment,
  conducting their normal job as decision makers.
  This was important as previous research have
  been done on real teams, but in these cases, the teams have not been in their
  organisational setting when doing the experiment of decision making.
  In the research conducted with teams in their normal work environment
  they have been set to do other team tasks than the ones they normally would do,
  thus the team work has not been truly authentic.

  To get such a team for the dissertation research,
  the president of the Stirling University Student Association (SUSA) was
  contacted asking if it was possible to use the SUSA executives board as
  the research group. This was approved by all the SUSA board
  executives and access was granted to attended all of the SUSA executive
  board meetings during the spring semester 2003.

  \section{Methods}
  Three methods were used in the research. The first method had obviously to
  be observation of the team as this was fundamental in order to answer
  the research question. For the same reason, it was necessary for
  the team members to take the BTRSPI, which was the second
  method used. In order to get more background
  knowledge about the team members and thus being able to get
  a profile of the executives, an interview with each board member
  was conducted as the third and last method.

  \subsection{The Belbin team role self perception inventory}
  The most fundamental element needed for this dissertation research apart from
  the team itself, is the BTRSPI. The BTRSPI material consists of three
  parts; the inventory itself,
  the analysis sheet for getting the results from the tests, 
  and a thorough description of the different Belbin team roles.

  In order to obtain the Belbin inventory, an appointment was made with the director
  at the University of Stirling Career Advisor's office.
  He was helpful both with providing the necessary material
  for the BTRSPI and training in how to conduct and analyse the test.

  The theory of Belbin's team roles is described
  in detail in section~\ref{btrspi}, so the text will now instead go on to give
  an account of how the BTRSPI itself is conducted.
  It should be noted that the BTRSPI does not test the ninth team role,
  the ``specialist''. This is due to the fact that it was first added
  to the Belbin team role model in \citet{rmb2} and the BTRSPI was
  published, without any later revisions, in \citet{rmb}.

  The inventory can be completed without any supervision necessary.
  Each participant is given a questionnaire that they are to complete. 
  The BTRSPI consists of seven sections which
  each have a statement about teamworking. The participant can choose between
  eight sentences that best match their self perceived behaviour in the team
  situation described. For each of these seven sections
  the participant is to distribute ten points among eight options. The participant may
  put all ten points on one answer, or distribute them among multiple options.

  The test is finished when the participant is finished distributing
  his/her points to the different sections and
  it is then ready for analysis. The key used for analysing the
  BTRSPI is called the ``Self-perception inventory analysis sheet'' (SPIAS) and
  was published together with the BTRSPI in \citet{rmb}. 

  % tkj a little cryptic. Describe in more detail.
  A problem that had to be solved before doing the observation of the team
  meetings was to not let the scores of the BTRSPI influence the observation.
  It is easy to let the results from the team members' BTRSPI influence the
  observation of the team meetings as one could unconsciously attribute people
  strengths and weaknesses according to their BTRSPI profile. To solve this
  problem, the BTRSPI was coordinated by the SUSA welfare officer, Helen Batty.
  This was possible because the conduction of the BTRSPI does not need
  any supervision, something which is one of its strengths and reasons
  behind its popularity among UK managers~\cite[33]{sbab}.

  The SUSA welfare officer handled both the distribution and
  collection of the BTRSPI to the executive board members.
  When returning the finished tests to the observer,
  a number indicated the different participants on the BTRSPI test
  as the corresponding names were written elsewhere only known to
  the SUSA welfare officer.

  This way, the team's BTRSPI profile was known when doing the observation
  and at the same time, the individual board members' profiles were kept anonymous
  until all observation was finished. When the scores from the BTRSPI should
  be compared to the individual team members'
  behaviour in the meetings, a key with the identities of
  the BTRSPI participants was given by the SUSA welfare officer. This key held
  the names corresponding to the numbers on the BTRSPI tests that the observer
  had gotten during before the observation period and it was then possible to
  compare the board members' Belbin profile with their behaviour in the
  team meetings.

  \subsection{The observation}
  The observation of the meetings was done by logging as much as possible of
  what was said and expressed during the meetings. Ultimately, it would have been
  desirable to video tape the meetings. However, this was rejected as this
  required further approval by SUSA and it was also potentially violating
  with the SUSA constitution.

  The recording of what happened during the meetings followed a scheme with a paper
  with a physical illustration of where the different members were seated.

  % tkj Reference to this. Connect the background with the practice.
  This way it could be commented on how the different
  groupings (if any) were situated in relation to each other and how people voted
  as a consequence of the seating in the room. It was interesting to see
  if members with opposing opinions had a placement in the room that resembled
  the opposition.

  The SUSA executive meetings were conducted as normal meetings are, around a big
  table where no one had fixed seating. The observer sat himself in
  such a way that that he could see as much as possible of what was going on during
  the meeting, including the body language of all members of the executive
  board during the meetings.

  When a person said something or used body language, it was noted down with a
  counting number. This way, it was possible to backtrack the meetings by following
  the running numbers back and forth between the different team members.
  Furthermore, at the beginning of the notes, the current topic was noted, so that it was
  easier to interpreting the meeting transcript afterwards.

  After each meeting, the transcript with the notes were entered into the computer
  and all members got all their meeting interactions listed under their
  name. This way, it was easier to do statistic analysis later when
  assessing the degree to what it was possible to predict
  the team members' behaviour from the BTRSPI scores. 

  An example of the finished transcript for one board member could
  look like the listing below. Each board member would have a
  similar block on the meeting transcript, giving a structured
  log of the meeting discussions.~\footnote{This transcript is only an example. Out of consideration of SUSA the
  different meeting transcripts are kept confidential.}:

  \begin{verbatim}
    John:
    5)  Orderly fashion presents a new case
        (wage increase for elderly).
    7)  Harder against Josh, faster
    11) Quite intense, defending
    28) Feels *strongly* for the case. Emphasises once again.
        Wants to prevent damage on SUSA.
  \end{verbatim}

  In this example, we can read that that the first interaction
  in the meeting by John, was as the fifth speaker in the meeting, because
  ``5)''  is the first item on the list of John's interactions. 
  John presented a new topic for discussion.
  For understanding the next interaction in the meeting, it
  is necessary to look at what is written under point number six
  on the meeting transcript. This will probably be
  under Josh's notes, as we can see that John
  went hard on Josh and started to intensify his speech at this
  point of time in the meeting (item ``7)'').
  Again, after some other executives have spoken, John
  continues intensively to defend his case. Lastly, as the
  \begin{math} 28^{th} \end{math} speaker, after a lot of other members
  of the board have given input to the case, John again claims his case,
  this time with very strong emphasis.

  By reading the meeting transcript and following the numbers, it is possible
  to follow get an adequate picture of how the meetings were conducted.

  \subsection{The interview}
  % Only use the interview as a means to get background knowledge about
  % the team; what the executives have previously done together, making
  % it easier to follow the meetings were references are being made back
  % to previous cases. Furthermore, the interviews gave substance to
  % the validity of the SUSA executive board as a real decision making
  % team in a genuine organisational setting ( the validity of the team ).
  %
  % Can also describe what I was initially writing about learning about
  % team members' behaviour, and why I didn't use it in the end.
  % Thanks Christina!

  During the two month observation period, the SUSA executive board members were
  interviewed about their own experience in organisational work,
  decision making training, team training, how
  they saw the group and who they were sociable with in the team. The
  question sheet used during the interviews is included in Appendix \ref{interview}.
  The interview helped creating a greater picture of the team than would
  otherwise have been given through only the BTRSPI and the observations.
  As mentioned earlier, the observation period was only for a short period of
  time and the interviews helped categorising the observed behaviour
  in the meetings were only exceptions to the norm or if they corresponded
  to the person's normal tendencies.

  By interviewing the executive board members,
  it was possible to get a broader picture of how the
  team used to behave (according to its members) and team-up in topics
  of conflict within the group. The background knowledge on these issues
  within the group would otherwise taken a whole year of observation to obtain.

  % tkj check again.
  By building a better picture of the group, the interviews also helped
  to give more substance to the team in terms of its validity as a
  serious organisational board. Although not very likely, it could be
  the case that the board was not very operative or active as a decision
  making team at the top of the SUSA organisation. If this were the case, it
  would have been very hard for the board to camouflage this
  because of the interviews were done with all the different board members
  individually.

  If the board had been a not serious one, it would have
  showed through the interviews as the range of questions were broad
  enough to get a measure of the scope and operation of the SUSA
  executive board. Therefore, the interviews gave proof of
  the validity of the SUSA executive board as a real decision making
  team.

  Furthermore, the interviews gave a profile of the team members, which can be
  viewed in Appendix \ref{susaprofile}. The profile gave useful background
  information about the board before going into the meetings for conducting
  the observation. Knowing some background history about the team and
  its members made it easier to relate to what was going on in the
  board meetings.

  The drawback of this data from the interviews, 
  is, of course, that it was not objectively gathered
  information, but given subjectively from the different executives.
  Because of this, not too much emphasis was put on these emotions
  when doing the analysis, they merely formed a
  basis for understanding the group.

  The interviews were done in an office where no one else could hear what was said.
  This way, the interviewee could speak more freely than otherwise if the interview
  was conducted in an open room were some of the other members of the team were
  present. This also made it easier for the interviewee to give an account
  inter-personal relationships and conflicts that might be present in the group.
  The questions were about the person's previous experience related to
  organisational work and especially instances were there was a similar setting
  of a decision making team. Furthermore, the interview contained questions
  about decision making training, team building and organisational training.
  These first part of questions were to give an understanding of the interviewee's
  experience in decision making in an organisational setting.

  The second part of the interview was to map the person's opinion about the team.
  He/she was asked how it worked in general, if it as a group had received training
  on how to conduct meetings and team work. Questions were then given about
  topics and decisions the team had dealt with the last year that the interviewee
  thought were particularly important, well handled and that were badly handled.
  The person was to explain why it went well or bad and how the team as 
  decision makers handled the particular case.

  Lastly, the interviewees were asked to give any thoughts on possible improvement
  of the team if they had more training and if they were to change any procedures
  in order to make the team meetings more effective.

  \section{Implementation problems and difficulties~\label{difficulties}}
  The initial plan was to do the research at Opera Software ASA, a middle
  sized computer firm making that is the biggest competitor to
  Microsoft on the web browser market. Unfortunately, this was not feasible.

  It was also attempted to cooperate with the Department of Psychology at
  the University of Stirling. This department has an interesting
  course unit where they do observation of teams,
  how they behave and what kind of roles the different
  team members take. These teams could have been used for the
  dissertation research. Unfortunately, the unit in which these team
  observations occur only runs in the autumn semester
  and was therefore not an option.

  The team interviews also caused some problems. The initial plan was to
  have three interviews with each member. One interview before starting
  the observation period, one in
  the middle of the observation period and a last interview
  after all the executive board meetings were finished.
  However, this was not possible because of the board
  members' schedules and hence availability. It was therefore sensible to have
  one interview with each board member that could be taken any time over the three months
  of interaction with SUSA. This way, people could take the interview whenever
  it suited them best.

  The greatest problem of the dissertation research was perhaps multitasking.
  The SUSA executives had meetings during the semester and not in the
  summer. Therefore, the research had to be done while following my own 
  units and submitting my course work. This was at times a tremendous workload
  and it is something that is not recommended for future research.

  %chapter 4
  \chapter{Results}
  This chapter will first present the results of two of the methods,
  namely the interviews and the BTRSPI.
  The findings when linking the BTRSPI scores with the
  observation of the team's behaviour in the board meetings will then
  be presented, before assessing the significance of these findings.

  \section{The SUSA executive BTRSPI profile}
  The first element of the dissertation research was the BTRSPI.
  The eight different Belbin team roles tested in the inventory are:
  Implementer (I), Completer (C), Shaper (SH), Plant (PL),
  Resource Investigator (RI), Monitor-Evaluator (ME), Teamworker (TW)
  and Completer-Finisher (CF)~\citep{rmb}.

  % tkj one paragraph describing the table.
  % Guide the reader through the table.
  The numbers in the first vertical column corresponds to
  the different SUSA executive board members taking the test.
  The letters on the top of the head row correspond to the different
  Belbin team roles, as described in section \ref{btrspi}.
  Thus, ``I'' is short for ``Implementer'', ``C'' stands for ``Completer'',
  ``SH'' is ``Shaper'' and so on. By reading each row, it is possible
  to get the BTRSPI profile of each participant.

  For example, row two in this table, starting with ``1'',
  is the Belbin profile of participant number one on the executive board.
  This person has, as most participants have, values for
  all team roles. That means that the person has some of the qualities of the
  different team roles. Even so, there are three scores that stick out,
  namely the scores for ``Implementer'', ``Teamworker''
  and ``Completer-Finisher''. These scores are the most interesting as they
  represent the person's team roles according to the Belbin team role model.
  To highlight this, the highest scores are given bold face to make
  it easier to spot the different team roles on the SUSA executive board.
  
  If a participant has one score that sticks out, i.e. has significantly
  higher score than the other values, the person fits perfectly with
  one of Belbin's team roles. However, it is common that a person has
  two roles, or even three roles, that he or she falls into,
  such as in this example.

  This phenomenon that a person falls into
  several team roles is nevertheless in accordance with Belbin's
  team role model. Belbin argues that a team does not have to consist of
  eight members in order to have all the eight team roles present.
  A person can hold several team roles contributing to a balanced
  team~\citep{rmb}.

  \begin{table}
    \begin{center}  
      \begin{tabular*}{1.0\textwidth}
        {@{\extracolsep{\fill}}ccccccccr}
        \hline  
        \hline  
        \bf{\#} & \bf{I} & \bf{C} & \bf{SH} & \bf{PL} & \bf{RI} & \bf{ME} & \bf{TW} & \bf{CF}         \\
        \hline  
        \hline  
        1 & \bf{14}& 6      & 4          & 6            & 8       & 7       & \bf{12} & \bf{13} \\
        \hline 
        2 & 7      & 8      & -          & 11           & 15      & -       & \bf{27} & 2       \\
        \hline    
        3 & 3      & 7      & 12         & 7            & \bf{25} & 8       & 4       & 2       \\
        \hline    
        4* & 7     & -      & \bf{15}    & -            & 4       & \bf{17} & 5       & 12      \\
        \hline    
        5* & 9     & 11     & \bf{16}    & 2            & 4       & 7       & 5       & 6       \\
        \hline     
        6 & 5.65   & 10.05  & \bf{13.25} & \bf{13.5}    & 9.55    & 8.5     & 6.3     & 3.2     \\
        \hline     
        7* & 13    & 12     & 13         & 1            & 2       & 2       & 14      & 13      \\
        \hline     
        8 & 6      & \bf{16}& 8          & 3            & 8       & 13      & 10      & 6       \\
        \hline     
        9 & 11     & 5      & 11         & 7            & \bf{13} & 11      & 6       & 6       \\
        \hline     
        10 & \bf{14}& 10     & 5          & 7            & 6       & 5       & \bf{13} & 10      \\
        \hline     
        11* & \bf{10}& 7     & \bf{11}    & 7            & 8       & 8       & 2       & 7       \\
        \hline     
        12 & 6      & 5      & 11         & 7            & \bf{16} & 6       & 19      & -       \\ 
        \hline 
        \hline
      \end{tabular*}
    \end{center}
    \small{*) Did not answer section I on the BTRSPI.
   The total scores are therefore lower than the other totals.}
    \caption{The SUSA executives' BTRSPI profile}
  \end{table}

  \newpage
  \section{Interviews}
  A lot of the board members had previous experience with organisation work
  at the university, either as members on the University Council or as executives
  on the SUSA board. However, additional organisational experience was scarce and
  there were only a few significant exceptions.

  Thus, training in organisational work and decision making in teams were
  usually limited to the course provided by the National Union of Students
  to the SUSA executives at the start of each university year.

  Most interviewees replied that they thought the executive board worked well as
  a team and that they had very well defined formal roles within the group, which
  the other members respected. This clearly separation of roles helped the
  group process, as described by \citet{rb}.

  For the most part, board members thought their contribution
  to the meetings was satisfactory, although some executives replied
  that they could have done a lot more as board members.

  However, it was the question ``would the team would work better if they had more
  training'' that created the biggest gap in the answers. The interviewees
  gave answers from both ends of the scale, from ``definitely yes''
  to ``absolutely not''.

  More unison were the interviewees when it came to the need of new procedures
  to the board meetings. Only a few suggestions came up and no one were essential
  for the effectiveness of the team meetings.

  % tkj find a way to justify the interview.
  % Hint from Christina;
  % There was a general tendency that the team members' behaviour in
  % the board meetings could be linked to what was described by
  % the members themselves during the interviews. 

  The interviews also gave hints of groups within the group and the social
  relationships between the different team members.
  There was a general tendency that the team members' behaviour in
  the board meetings could be linked to what was described by
  the members themselves during the interviews. However, because of
  the biased nature of the interviews, nothing could be concluded from them.

  Nevertheless, the interviews confirmed that the team of SUSA executives itself
  was a genuine decision making team that had
  true influence on the organisation it operates within. This was the primary
  objective of the interviews.

  \section{Findings \label{findings}}
  There was a general tendency that the team members' behaviour in
  the board meetings could be linked directly to their Belbin team role
  profile.

  Approximately two thirds of the interactions in
  the meeting room could be explained by either the characteristics of a team role,
  the role's strengths, or its weaknesses. Furthermore, it was interesting to
  see how people took opposing stands in cases and how well this related to
  Belbin's description of ``potential conflicts'' and who the team
  roles ``relates well with''. It was found that the BTRSPI 
  drew a believable picture of how the team members would ``team up''
  in topics of dispute.

  %chapter 5
  \chapter{Discussion}
  % discussion of the findings including their scope and possible limitations.
  \section{Summary of key findings}
  It is very interesting to look at the BTRSPI scores for the SUSA executive board.
  The team appears to be balanced with all of
  Belbin's team roles present. However, there are for example four shapers in
  the team, something which according to \citet{jpns} can be disruptive and lead
  to conflict in the team.

  Indeed, this could be observed in the meeting room where
  the discussions could be very emotional at times.
  Despite of this fact, the team seemed to handle this disruption reasonably well
  and it was not an unbearable number of conflicts during the meetings. A little
  tension is positive and should be seen as something constructive
  as long as ideas and not personalities are the sources of
  conflict~\citep{cg}. Therefore, it would be an exaggeration to say that the number
  of shapers ruined the team dynamics of the SUSA executive board.

  In most cases the Belbin profile of the team worked well to explain who
  teamed up in arguments. It was a clear tendency that team roles that
  ``related well together'' according to Belbin did cooperate well together in the
  meeting room.

  The deviation in accuracy of the Belbin scores was for the most part related to
  individual behaviour and not how individuals related to each other. For example,
  it was found that participants that according to themselves where both a shaper
  and a plant, were totally silent during meetings, something which is a clear
  contradiction to the Belbin theory. On the other hand, when it came to the in-group
  relationships, such as who worked well together and who were natural deviates,
  the Belbin scores did well to predict a team's behaviour. Thus, if only
  the results of what team members related well together and who did not
  during the executive board meetings were to be counted, the
  BTRSPI accuracy would get a higher score. However, as the research
  was set out to measure how well the BTRSPI could predict the behaviour of the
  the team members during their meetings, the BTRSPI accuracy was ``only'' 67\%.

  \section{Significance of findings}
  % and comparisons to literature review and research objectives.
  The findings have significance to the discussion of the validity of
  the BTRSPI, especially in connection with the applicability of the inventory
  to decision making teams in an organisational context. This research supports
  the study of \citet{sfthwm3} who concluded that the BTRSPI is suitable for
  non-managers also.

  The point to which this study has the greatest significance, is the
  use of a real team in a real setting as comparator to the BTRSPI profile
  of the team. There has been very few attempts to do this as pointed
  out by \citet{dphh}. A rare exception to this fact is the research
  conducted by \citet{bsss}. This research by \citet{bsss} differs from this
  dissertation because the authors did not do the
  observation themselves. The study was conducted by
  peer review and carries the drawback that the observation
  may be influenced by personal opinions about the work colleagues. For this
  reason, the observation is better conducted by a third party who does
  not have any relation to the team and can thus be more objective,
  as was the case in the dissertation research.

  The dissertation confirms what \citet[11]{cj} and \citet[652]{jpns}
  concluded; a balanced team with the presence of all the different team
  roles has a positive increase in performance.
  This was also confirmed by the members of the SUSA executive board through
  the interviews. All members of the board believed
  that the team performed well as a decision making body. Furthermore, the
  list of accomplished tasks throughout the year is impressive enough to reach
  the conclusion that the executive board worked well as a decision making
  team. In addition to this, the observation of the executive board a
  number of meetings gave proof of that the team does function well and
  represented a balanced decision making team.

  The dissertation research question (\emph{Is it possible to predict the behaviour of a decision making team from the team role profile given by Belbin's team role self perception test?}) was given a positive answer from this research. In more than
  two thirds (67\%) of the cases, the BTRSPI predicted the behaviour of the team
  members.

  All the same, it could be argued that this is not a definite answer.
  First of all because the positive proportion was ``only'' 2/3.
  Some critics would argue that one third (32\%) of the
  the data claim otherwise is a big enough proportion to answer ``no'' to
  the dissertation question.

  The answer to the high inaccuracy of the BTRSPI may lie in the self perception
  nature of the inventory. As it is conducted by the participants themselves,
  it is possible that some participants do not have the ability to be objective enough
  to fill out the BTRSPI with the right amount of self criticism.

  From a different perspective, the dissertation findings
  support the likes of \citet{sfwmgs} and \citet{sfwmjw}
  who suggest that although the BTRSPI does carry weight, it should not be used as
  the only measure of team composition, team success or team building. 
  The observation of the SUSA executives revealed inadequacies in
  the BTRSPI by discovering contradicting behaviour by some team members in
  the board meetings. Therefore, although the observation did find proof of validity
  for the BTRSPI, it also found the opposite, proving that an additional measure
  like itself is need in order to get a picture of a team's profile.

  Nevertheless, it is important to view this research in
  the light of the fact that the
  research period was not long enough to conclude anything certain, although a clear
  tendency could be found. The executive board should have been followed for a
  year in order to get a broader basis for data analysis. Furthermore, more interviews
  at different stages with the team members would be preferable. As mentioned in
  section \ref{difficulties}, this was not doable in this research. 

  All the same, it is safe to argue that this research does give evidence
  of the validity of the BTRSPI as also concluded by the
  research of e.g. \citet{sfthwm3}, \citet{sfwmjw},
  \citet{sstm}, \citet{bwmg}, \citet{ysht} and \citet{jssd}.

  \section{Implications of findings for human resource management and SUSA}
  Human resource management (HRM) is a concept with a wide range of elements,
  often dependent on what the firm allows the human resource manager to be,
  therefore HRM can be both ``soft'', ``hard'' and everything
  in between~\citep{pthr,jhjl}. ``Soft'' HRM relates to an HRM strategy where
  management wants to develop the human assets in the company and at the
  same time ensure a high degree of commitment from the employees through
  emphasis on communication, motivation and leadership. ``Hard'' HRM on the other hand,
  relates to a strategy which focuses on the size and cost of the employees, through
  an emphasise on rationality, strategy and business
  needs (Storey, cited in~\citet[47-48]{jhjl}).

  In addition to the soft-hard spectrum, HRM consists of a number of
  other building blocks, such as; employee law, company culture, national culture,
  industrial relations, interview technique, motivation theory,
  industrial democracy, quality circles and team
  building~\citep{ahdb,sr,jhbm}.

  Therefore, teamworking is a central part of
  human resource management~\citep{fzje} and
  the BTRSPI is a central part of management's toolbox to measure
  team composition and effectiveness~\cite[697]{dphh}.
  A useful discussion and extensive range
  of research on the validity of BTRSPI is therefore important. This dissertation
  is a contribution to this discussion as it gives a new attribute to the
  previous research by linking the BTRSPI to the behaviour of a real
  decision making team in a real, organisational setting.

  For the team that has been the focus of this research, the SUSA executive
  board, this dissertation can be useful reading as well, especially
  when looking at the self perception aspect of the test. Some of the
  participants on the board had so strong deviation from their BTRSPI scores and their
  actual behaviour in the meetings that two questions
  could be raised; were the participants in question totally honest
  when assessing their own role as a team members,
  and if yes, why do they not behave in accordance to their team roles?
  Is there anything in the meeting procedures that prevents them
  from acting as their true selves, or is there any other team members
  that obstruct them from being contributing members of the executive board?

  These questions may give valuable answers that, when assessed, can
  help improve an already well working decision making team of
  SUSA executive board members.

  \section{Recommendations for future research}
  The primary recommendations for people wanted to explore further along this
  topic, is to use more time. Naturally, since this was ``only'' a MSc dissertation,
  the time available for data collection was limited. In order to get a greater
  foundation and greater insight knowledge, it would be desirable to follow the
  executives their whole year of operation and not only for two months.

  People may alter their behaviour after the topic discussed. In this sample, each
  member has their own area of expertise, e.g. entertainment, and will naturally
  be more engaged in discussions relating to their field. By following the team
  through a whole year, one would get a better rounded picture of the team's
  effectiveness as decision makers as the behavioural patterns would be more
  evident to the observer.

  Another point to consider about following the executives through a whole
  year is related to the nature of this kind of team. As SUSA is an organisation
  for students, it is likely that the demands and pressure will vary on the executives
  through the year in accordance with the university cycle. Moreover, the motivation may
  be greater among the team members at the start of their period than in the end of
  the second semester. However, it is important to stress that most members of the
  executive board had been on it for several years and had by that gained
  valuable experience in organisational work and team decision making.
  
  Nevertheless, it was possible to pick up a general trend in the
  board meetings from the time available for the dissertation data collection.

  %chapter 6
  \chapter{Conclusions}
  The central topic of this dissertation has been the validity of the Belbin
  team role self perception inventory (BTRSPI).
  It contributes to the vast discussion of the
  validity and applicability of the BTRSPI by observing a real decision making team
  and linking this to the team's BTRSPI profile.

  The current literature on teamworking and Belbin's team
  role self perception inventory
  has grown to be quite extensive. Even so, when going through the different research
  projects, it appeared evident that little research has been done
  on real teams in connection with the Belbin inventory.

  The main themes that run through the previous research fall into three categories.
  The first is to look at one industry or company at large and assess
  the impact that teamworking has on the nature of work; on
  labour turnover, worker output, absenteeism and job satisfaction.
  Another perspective is to map the tendencies of what kind of
  BTRSPI profile different
  groups of people have, such as what kind of BTRSPI profile managers or law students
  tend to have. The last category of research is to explore
  the BTRSPI and team behaviour. However, this is for the most part done on student
  teams, solely composed for the research projects. This has the effect that
  what goes on in the teams will not have consequences afterwards, hence an
  important aspect of teamwork is lost, namely the social relationships. People do not
  want to be unpopular with people they have to work with on a daily basis.
  When the teamwork is only for a short time period and/or with a task
  that does not affect their daily routine, such as in much of the research
  mentioned, people tend to be more forgivable and act in a more
  rational manner than they otherwise would have in a real setting.

  This dissertation focused on how one decision making team behaved in their genuine
  environment. The team chosen was the executive of the Student Association
  at the University of Stirling (SUSA). By comparing
  their behaviour in the meeting room
  with their BTRSPI profile it was found that indeed a lot could be explained by
  the characteristics of their Belbin profiles. Two aspects of the Belbin roles
  were emphasised, the characteristics of the different
  team roles (with strengths and weaknesses) and who the different
  roles would relate well with. By
  following each interaction of team meetings over a number of meetings,
  it seemed clear that it was possible to predict a great proportion of the
  team member behaviour in the meeting room. Or phrased differently,
  it was possible to anticipate the different team roles from observation in
  two thirds of the cases.

  Even so, there are some aspects to consider when assessing the
  significance of these results. First of all
  could the data basis for the research be broader as the
  time for conducting the observation was limited because the board meetings
  were during the semester and not during the summer. Secondly, it is important to
  note that since one third (32\%) of what happened in the meeting room was
  not predicted by the Belbin inventory, but on the contrary were
  total opposite behaviour of what was ascribed to the particular team role, it
  can be argued that the BTRSPI is not the definite answer.

  This theme brings us to the conclusions of this dissertation. Firstly, this
  dissertation supports \citet{sfthwm3} who argued that
  the BTRSPI is applicable to non manager teams also.
  The second conclusion is that the inventory carries adequate validity and by that
  the dissertation supports the likes of \citet{bwmg}.
  Nonetheless, perhaps the most important
  outcome of this dissertation is that it is important to remember that the
  BTRSPI should not be used as the only way of assessing team
  composition and performance, as also concluded by \citet{sfwmgs}.

  Therefore, this dissertation supports what \citet{rmb3} himself argued;
  the BTRSPI was never meant to be a self standing psychometric test.
  The Belbin team role self perception test is absolutely
  a worthwhile tool to use by management
  when assessing a team's composition and possible performance, as
  long as it is supplemented with additional measures.
   
  % appendix
  \newpage
  \appendix
  \setlength{\baselineskip}{1.3em}

  % appendix a
  \chapter{The interview \label{interview}}
  These are the questions used for the interviews with the SUSA executive
  board members. The questions are open ended in that the interviewee
  can elaborate to the extent he or she wishes.

  \section*{Questions about the interviewee}
  \begin{itemize}
  \item What is your previous experience in this kind of organisational work?
  \item Have you had any training in teamworking?
  \end{itemize}

  \section*{Questions about the team}
  \begin{itemize}
  \item How do you think the SUSA executive board works?
  \item Do you have/gotten any guidelines for meetings/teamworking?
  \item How do you contribute to the team?
  \item Can you give give a listing of the most important cases last year, that
    the board has dealt with?
  \item Can you give examples of any crucial decision that the team
    has dealt with where you worked well and dealt with the issue successfully?
    \begin{itemize}
    \item What was it that worked well in that particular case?
    \end{itemize}
  \item Can you give an example of something that did not work so well?
    \begin{itemize}
    \item What was it that went wrong?
    \item Why did it go wrong?
    \end{itemize}
  \end{itemize}

  \section*{In hindsight}
  \begin{itemize}
  \item Would the team have been more effective if they had more training
    in teamworking or decision making?
  \item Would you recommend other practices for future
    generation of executive board members?
  \item Do you have suggestions for other meeting procedures,
    or how the SUSA as an organisation work?
  \end{itemize}
  
  \chapter{Interviewees' profile \label{susaprofile}}
  There was 6 male and 5 female members on the executive board. It should
  be noted that one additional exec was present as this person was to be on the
  executive board next year, but stepped in already this semester. All the members
  of the board was between the age of 20 and 24.

  All executives get training from the National Union of Students. This is a weekend
  course in which they learn how to run a meetings, team building, how to be an
  executive officer, specifics about the SUSA organisation and practical exercises
  as they would occur in the real meetings.
  
  \begin{table}[h]
    \begin{center}  
      \begin{tabular*}{1.0\textwidth}
        {@{\extracolsep{\fill}}cccr}
        \hline
        \# & Relevant experience & Sex \\
        \hline
        % Richard
        1 & Five years on the SUSA council, two years as an executive & M \\
        % Helen
        2 & --- & F \\
        % Declan
        3 & Fourth year on exec. Organisational experience from school & M \\
        % Kevin
        4 & Clubs and societies at schools, including political debates. & M \\
        % Esther
        5 & Third year of SUSA involvement. School councils \& committees & F \\
        % Nicola
        6 & Second year on exec. & F \\
        % Laura
        7 & --- & F \\
        % Sean
        8 & --- & M \\
        % Chris, exec next year
        9 & --- & M \\
        % Rachel
        10 & Third year on exec. Youth work. & F \\
        % Mark evans (sports union) 
        11 & Third year on exec  & M \\
        % Ian
        12 & Prefect at primary school.  & M \\
        \hline
      \end{tabular*}
    \end{center}
      \small{In the ``Sex'' column, ``M'' is male and ``F'' is female.}
  \end{table}

  \chapter{Meetings attended}
  It should be noted that meetings in April and May were moved around
  in order to meet other university arrangements such as meetings and exams,
  thus the frequency of meetings in May.
  The last meeting of the year was May the \begin{math} 15^{th} \end{math}.
  \begin{itemize}
  \item \begin{math}18^{th}\end{math} of February 2003
  \item \begin{math}24^{th}\end{math} of April 2003
  \item \begin{math}1^{st}\end{math} of May 2003
  \item \begin{math}6^{th}\end{math} of May 2003
  \item \begin{math}8^{th}\end{math} of May 2003
  \item \begin{math}13^{th}\end{math} of May 2003
  \item \begin{math}15^{th}\end{math} of May 2003
  \end{itemize}

  \newpage 
  \bibliography{paper}
  \bibliographystyle{plainnat}
  \nocite{*}

  \bigskip
  \begin{center}
    \rule{0.9\textwidth}{0.011em}\\
  \end{center}
  The dissertation was typeset with~\LaTeX~using 12pt sans serif font.
  Citations were made with BibTeX using the American Sociological Review
  citation style.

\end{document}


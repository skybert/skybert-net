\documentclass[a4paper,12pt,titlepage]{article}

\begin{titlepage}
  \title{Employee Relations~\& the Law \\ -- \\Essay Question 8}
  \author{Student number 1121320}
%  \date{November 13, 2002}
\end{titlepage}

\pagenumbering{Roman}
\pagestyle{plain} % plain is default
%\usepackage{natbib}
\usepackage{harvard}
%\bibpunct{(}{)}{;}{a}{,}{,}

\begin{document}
  \maketitle
  \setlength{\baselineskip}{2em}
  
  \section{Introduction}
  Unions may be very influential when first taking industrial action.
  London has experienced almost a standstill on two occasions this
  fall due to two 24 hours strikes by the London Underground Union.

  The background for this was, not surprisingly, the unions'
  claim for a pay raise. The London Underground proposed
  an increase of three percent to its employees. But, the unions
  claimed a 5.7 per cent increase, something which the London Underground
  thought was unheard and refused to reopen the negotiations. 

  These strikes have had an estimated cost of �120 million in lost
  productivity for the London economy. Although the London Underground
  only serves the capital itself, it has huge impact on the whole
  island. The London subway system carries more passengers
  every day than the joint work of all the major lines in England.
  By going on strike, the London Underground union to a
  large extent hindered London's international business and
  thereby crippled Britain.

  The strikes resulted in that the current major in London,
  Ken Livingston, promised the London Undeground's employees
  a pay increase when he takes over the subway next year.
  The unions then cancelled the strike~\cite{tt}.

  The whole of Europe experience a vast number of strikes
  in the industry in the period 1968-1974. The period is therefore
  called ``The Wave of Struggle''. The UK was of course no exception
  and the unions were so strong, or the situation was so bad, that
  the Labour government under prime minister Wilson, made in 1974
  a social contract with the unions, a manifesto that basically stated
  asked the union to ``play nice'' since the UK economy was dire straits.
  The situation was so serious, that the UK had to
  apply for a loan from the USA to keep its industry running.
  
  When the Thatcher government took the power in 1979, 
  they wanted to reduce the unions' power so that they could not
  affect the industry in that large extent has they had previously done
  during this ``wave of struggle''. This resulted in a number of
  changes to legislations regarding the unions.

  I will in this essay look at different changes in
  industrial relations legislation in Britain since 1979 and
  give a discussion what they have had to say
  for the trade unions' ability to take industrial action.

  \section{Industrial Relations Legislation 1979-1993}

  During the years of the Thatcher government, seven 
  separate acts were passed, both to create statuary immunity
  for the employers and for making it harder for the unions
  to take industrial action legally~\cite[2]{ids}. It have been
  argued that all trade union legislations between 1979 and
  1993 have been issued to cripple the trade unions and hinder them
  from functioning effectively by enforcing the unions to adopt a more
  traditional bureaucratic organisational hierarchy.
  In addition, the acts were to give the unions 
  several legal constraints, making it harder for them
  to take industrial action~\cite[301]{km}.
  
  The first employment act under the Thatcher government came in
  1980 and contained a number of changes that had an impact on
  the trade unions.

  First of all, the act introduced changes to the ballot issue.
  It became more extensive for the unions because all workers
  that they believed to be affected by a strike, had to
  participate in the ballot. That is, all workers, that the
  union believed could be taken out in strike, had to
  participate in the ballot. The ballot had to be 
  written, a show of hands was not good enough.
  This presented a timing problem, as the time was often
  crucial for the effect of an strike in response to the
  current state of the negotiations between management
  and the union officials~\cite[190-1]{bs}.

  Secondary picketing was another of the important elements
  of the 1980 Act. Secondary picketing was made unlawful
  and this took away a great weapon for the trade unions. Secondary
  action was for workers to do industrial action at a different
  employer's grounds in order to increase the pressure of the
  employer involved in the dispute.
  Secondary action was often a sympathetic gesture,
  a demonstration of support for the workers involved in the
  dispute. It became highly
  unpopular among the politicians and the government wanted
  to get rid of it because
  it spread the disputes and difficulties over the
  industry and community at large~\cite[183-184]{bs}.
  
  The same year came also the Social Security Act 1980 that basically
  made the financial situation for strikers less beneficial
  as compared to earlier years. The conservative government
  would not ``finance strikes'' as they put it, and thus cut the
  social benefits of striking workers. The striking workers then
  became more dependent on funds from their family or union support.
  It was in the government's belief that it was just fair that
  the union helped share the burden of striking, non producing
  workers that was put on society~\cite[107-108]{ke:trts}.

  In 1982 there came amendments to the Employment Protection
  (Consolidation) Act (EPCA) from 1978. The amendments
  contributed to giving the employers greater power. The
  employer could now dismiss all workers involved in a strike.
  This was in itself a very powerful weapon for the employer,
  especially in workplaces with unskilled labour and
  no shortage of unemployed labour in the area. If the workers
  went on strike over some dispute with management, the employer
  could simply fire the striking workers and hire new labour from the
  streets to replace the previous workers~\cite[44]{ke:trts}.

  This was just what happened in London two years after
  the act was introduced. Rupert Murdock was building a new
  production plant for his new paper the ``London Post'' in the
  London dock lands in 1985. The negotiations with the
  trade unions of working agreements for the workers to work
  in the new plant and also for existing Murdock employees
  at other sites to move to this new place Wapping went into a halt.
  The unions took out all workers in strike against Murdock. He
  replied by firing all 5500 workers since they were in breach of contract
  according to civil law, and hired new staff to replace the
  former paper plant workers~\cite[1-2]{ke:trts}.

  Furthermore, the 1982 amendments had several points that made
  it even easier for the employer. First of all, the employer would
  earlier lose his immunity if he re-engaged a dismissed worker. Now,
  the employer could re-engage a worker within three months after
  the dismissal without losing their immunity~\cite[254]{jmfs}.

  Another piece
  of the old legislation required the employer to dismiss all
  workers that were striking, even if some employees had
  returned to work and only some unionists were still on strike.
  This changed in the 1982 amendments, so that the employer
  could re-engage the workers that had come back to work
  voluntarily before the strike had been called off. This had
  great impact on the unions, and caused more problems than
  it first might seem. Firstly, it was crucial that all striking
  workers got the warning, that if they did not return to work
  they would get dismissed. Secondly, and maybe more importantly,
  what would happen in reality, was that the ones still on strike
  would be the activists, the members of the union with
  the greatest opposition to the management. Therefore, this
  became a great weapon for the employer, since he now could get
  rid of the trouble makers only without dismissing the rest
  of the workforce, giving the best from both worlds, seen
  from the management's point of view~\cite[45]{ke:trts}.

  In addition to this, the employer was given an additional
  beneficial piece of legislation in the 1982 amendments. 
  The employer could now choose to dismiss only one group of the
  workers. This was a significant contribution to the employer, since
  he now could fire a lot of the striking workers to signal that
  he did not give in to the trade union's pressure, but at the same
  time, he could keep the production going in the rest of the
  company. This made the down time significantly shorter for the
  company before they got installed new staff for the dismissed
  workers, because production could continue almost as normal
  in big parts of the firm~\cite[46]{ke:trts}.

  The Employment Act of 1982 imposed another constraints to the unions.
  This was that if the union members were indirectly hindering the
  supply of goods or services because the ones supplying this
  was non union members, the union lost its immunity and
  could be sued by the employer~\cite[188]{bs}.

  Another element of the act that empowered employers to sue
  the unions to a great extent, was that the unions were held 
  liable for damages caused by their members when taking industrial
  action~\cite[27]{ilw29}. This law had a huge impact on trade unions,
  as company financial claims were heavy blows to the union organisation.
  The union was only liable for acts authorised or endorsed by
  a responsible person, that is a person with in the union
  that had the union authority to act on the union's behalf, e.g.
  a shop steward~\cite[173]{bs}. 

  With this liability, there came also another law that put
  limits to the financial claims of the employer. A limit was
  set of how much money the employer could sue the union for.
  This was to reduce the chance of the unions going bankrupt because
  of financial claims from the firms.
  Furthermore, the 1982 Employment Act gave the union some
  protected property that financial claims could enforce money
  from. These properties included the union's
  political fund, if any, and the union repositories for
  paying sickness, accident, death, superannuation and the
  like~\cite[174-175]{bs}.

  

  Maybe the biggest issue in the 1984 Trade Union Act was the one
  of ballots. First of all, almost any action by the union without
  a ballot was considered unlawful~\cite[183]{bs}. The only action
  by the union that was not unlawful without a ballot was the
  threat of going to strike~\cite[189]{bs}. The unions
  were required to do secret ballots for electing their members,
  mastering political funds and before calling any industrial
  action~\cite[298-299]{km}. An industrial action without
  a ballot was considered unlawful and the unions therefore had
  to spend valuable time collecting the ballots.

  The TUA of 1984 did not put any requirements on the employer to
  co-operate in conducting the ballots, not even a note of
  letting the union officials in on the company premises to
  hand out the ballot fliers to the workers were included in
  the act. The 1980 Employment on the other hand, did
  include a request for employers to provide facilities
  for the ballot to be conducted. However this law has two important
  restrictions. The union has to be recognised by the firm and
  the company needs to employ more than twenty
  workers~\cite[311]{ke:tuap}.
  
  Another thing concerning the ballots and maybe the most
  controversial element of the Trade Union Act 1984 was
  that unions were required state on their ballots that the
  worker, if going on a strike, would  be a breach of
  his or her contract~\cite[267]{jmfs}. 
  Without this statement the ballot was not to be considered
  valid. The condition was to be phrased as a question and the
  worker was to answer ``yes'' or ``no'' to whether he was ready
  to take part in a strike if so was demanded or another
  industrial action where no strike was involved.
  This statement had a huge impact on the union officials work, because it
  made it more difficult to convince worker to stand behind the union and
  go on strike if needed. It was also difficult to formulate
  a justification on a flier to why the worker
  needed to say ``yes'' to this question and be ready to commit
  a breach of his or her contract~\cite[191]{bs}. 

  The 1986 Wage Act did require the employer to continue to
  pay wages to suspended workers. However, this did not apply
  to striking workers, and did therefore not help the
  unions~\cite[26-27]{ke:trts}. 
  
  The 1988 Employment Act also included a law that state that
  the trade union could no longer discipline its members for
  not supporting the industrial action. Together, with the
  hurdles regarding the conduction of the ballot itself, this
  made it even harder for the unions to get the support
  they needed among the workers~\cite{de}.
  
  Even though both the 1980 and 1982 Employment Acts had
  tightened the laws on closed shops, it was first in 1990
  they were totally removed by law. The act removed any
  compulsory element of ``closed shop'' from union
  membership. This took away an important
  reason for many employees to take union membership. Previously,
  sometimes the only reason for employees to join a union was because
  they really did not have a choice. With the ``closed shop'' element
  removed from union membership, there was one less reason to
  be a union member, and in some cases this was
  enough~\cite[297]{mnpb}.
  
  In 1992, all law concerning industrial action were put into they
  ``Trade Union and Labour Relations (Consolidation) Act'', TULRCA. 
  
  One requirement put on the unions in the TULR(C) Act 1992, was
  that when they were balloting their members, they needed to
  get separate approval for strike action and for industrial
  action short of strike. That means that they first needed to
  know how many of their members were in favour of a strike, and
  then separately, they needed to know how many members that were
  in favour of an industrial action against the employer, that was
  not a strike~\cite[8]{ids}.

  An amendment to the Trade Union and Labour Relations (Consolidation)
  Act came in 1999 and regarded the ballot. The change was that
  the unions no longer needed to give the management any names
  of persons they want to ballot~\cite[294]{ilw28}.

  The TULRCA also stated that a union could not conduct a
  ballot after an industrial action in order to support it. As
  stated in the 1982 Employment Act, the union will receive the
  responsibility for an industrial action conducted by a ``responsible
  person'', but they will not keep their immunity in tort.
  The problem arises when there is an unofficial industrial action
  that is not endorsed by the union, but the union sympathise with
  it and want to turn the action into an official one. This
  is no longer possible with the TULRC act.
  The union is left in a lose-lose situation, with
  this law, because if they do not take the responsibility for
  the action, the employees that conducted the industrial action
  stands without protection and cannot claim an unfair dismissal, and
  if the union do take the responsibility, they lose their immunity in
  tort~\cite[14]{ids}.
  
  \subsection{Industrial Relations Legislation 1993-1999}
  After a significant number of acts that were primarily to
  the benefit of the employer, 1999 saw the light of a bill
  that was a great leap in the right direction from the
  workers' point of view. The bill was a great attempt to close
  the gap between Britain and the ILO of 87 and 98. The ILO 87 and 98
  give the workers in the EU far greater protection than
  what has been and still is, the case in the UK~\cite[283]{ilw28}.

  Blacklisting is the first of the important elements
  of the 1999 act. Employers are no longer allowed to compile
  lists or indexes of union members. Both making these lists and
  distributing them have become unlawful by this act~\cite[285]{ilw28}.

  Another, significant development empowered by this act is 
  the right for a worker to be accompanied by a trade union official. This
  applies when the employee is asked by the employer to
  attend a disciplinary or grievance hearing. The employee can
  then choose one person to accompany him to the hearing. The
  person accompanying the worker can either by another worker
  of the employer or a trade union official. The person
  is not to be lawyer, if not the fellow employee or
  trade union official happens to be a lawyer by
  coincidence~\cite[16]{ilw29}. However, the law does no to apply to 
  to negotiations between the employee and the employer regarding
  higher pay, shorter hours or increased overtime~\cite[289]{ilw28}.

  The Employment Relation Act 1999 has also a protection
  from dismissal upon strike.
  If the worker is dismissed because he took protected industrial action,
  the worker will be regarded as unfairly dismissed. Protected
  industrial action means that the action must be supported by
  a valid ballot as stated in the Trade Union Act
  1984~\cite[18]{ilw29}. However, this
  protection does not apply to other workers than the trade union
  officials themselves. The unions as whole, are therefore not any closer
  to getting the ``right to strike'' to be a legal
  action~\cite[292]{ilw28}.

  Ten different aspects to the balloting law has been amended in the
  1999 act. One of these aspects is that
  a ballot is no longer invalidated because the union failed to
  notify the employer of the result of the ballot. On the other side,
  the union is required to both give the employer information about
  the coming strike so that the employer can make necessary
  preparations, e.g. notifying the customers of the company of the coming
  disruption of work. The unions were also now required to give the
  employer a certain amount of time notice before the initiation of
  the strike~\cite[294-295]{ilw28}.

  However, the unions did not have to give the names of the
  employees they wished to ballot. This was a good change from the
  Trade Union and Labour Relations (Consolidation) Act 1992, since
  this act stated that the employer could force the union into court
  in order to give up names of the persons involved in
  the ballot. After the 1999 act, this changed and the union was only
  required to give up details and numbers
  to the employer~\cite[17]{ilw29}.

  The 1999 act also introduced a scheme of obligatory
  recognition of trade unions. All companies that have
  more than 21 employees at the day of request for recognition by
  the union, must recognise the union, if they do not already have
  recognised a union at their company~\cite[33]{ilw29}.
  Therefore, there is now place in business now, where the
  unions can lawfully be excluded~\cite[290]{ilw28}.

  \section{Conclusion}
  The trade unions way of operation got a drastical change 
  after 1979 when Thatcher took power. The conservative
  government passed seven separate acts in their reign, and
  the general tendency was clear: it became more and more
  difficult for workers to take industrial action~\cite[243]{jmfs}.

  When assessing the unions' ability to take industrial action, there
  is one outstanding point, that industrial action is unlawful.
  Basically every industrial action except the refusal to
  work longer hours than what is stated in the employment contract,
  is a breach of contract~\cite[9-14]{ke:trts}.

  So by participating in industrial action, the individual worker
  must be ready to face unemployment if his or her employer
  chooses do to dismiss the workers as a reaction to the strike. The 1980
  and 1982 Employment Acts give the employer full right to not
  only fire the workers that are striking, but also to strategically
  select the workers that are to be fired. This way, the employer
  may keep the business going while seeking new staff. Furthermore,
  the legislations introduced in the 1980s forced a more
  bureaucratic mode of operation on the union, making it into a slower
  moving organism~\cite[301]{km}.
  The acts passed in the 1980s also made the
  process of taking industrial action more time consuming for the
  unions, making industrial action a less powerful action since
  it lost its timing.

  Together with the National Minimum Wage Act 1998
  and Working Time Regulation, The Employment Relation Act 1999
  must be seen as a positive regulatory~\cite[12]{ilw29}.
  The 1999 act made it an unfair dismissal to fire a person because
  the worker participated in a lawful strike. However, when this
  only applies to the union officials and not all the workers, the
  law is nothing more than a small step in the right direction, the
  basic principles remain the same. This may not come as any surprise,
  as it is what the current (1999) Labour government
  promised, to not reverse the fundamental legislation
  introduced in the Thatcher period~\cite[28]{ilw29}.

  Regardless of all the legislation that has been made in the
  90s to protect the individual worker, the bottom line is that
  the individual worker in today's Britain is almost without
  protection of any kind, and most ironic of all, taking industrial
  action is unlawful. 
  Collective bargaining have been the
  way workers have negotiated better deals with employers in
  Britain, but as long as the workers' only real weapon, industrial
  action is an unlawful one, how can this system work? Or
  as Lord Wright put it:
  \begin{quote}
    The right of workmen to strike is an essential
    element in the principle of collective bargaining
  \end{quote}
  But without a lawful right to strike, how can there be collective
  bargaining?~\cite[41]{ke:trts}.

  The bottom line is that the
  union's ability to take industrial power have in general
  been weakened by the industrial legislations after 1979, and
  it is not likely that there will be drastic changes
  to the union's current state since the Labour party will
  not attempt to reverse the conservative party's legislations
  during the 1979-1993 period, regardless of advancements in
  new industrial legislation like the National Minimum Wage Act 1998
  and the Employment Relation Act 1999.

  \newpage
  \setlength{\baselineskip}{1.3em}
  \bibliography{essay}
  \bibliographystyle{jmr}
  \nocite{*}

\end{document}
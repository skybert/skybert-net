\documentclass[a4paper,12pt,titlepage]{article}

\begin{titlepage}
  \title{Management Development \\ -- \\Essay Question 3}
  \author{Student Number 1121320}
%  \date{April 14, 2003}
\end{titlepage}

\pagenumbering{Roman}
\pagestyle{plain} % plain is default
\usepackage{harvard}

\begin{document}
  \setlength{\baselineskip}{2em}
  \maketitle
  
  \section{Introduction}
  Henry Ford is famous for being the first to
  fully implement Fredrick Winslow Taylor's principles of
  scientific management, leading to Ford being the leading car
  manufacturer in the world. However, the downside of his success
  was that deskilling of labour and scientific management
  of the labour process lead to extreme alienation of the employees
  resulting in job dissatisfaction amongst the workers~\cite{hbe}.
  This was also a part of Harry Braverman's criticism of
  capitalism's industrial development, commenting on how deskilling of 
  labour and scientific management of the labour process made the workers
  alienated to their workplace~\cite{hb}.
  
  This was definitely one of the reasons
  why Ford introduced teamworking as a part of the corporate strategy
  in the mid 1980s when they tried out total quality management. 
  The organisation of Ford then started using teams as the centre of
  its organism~\cite[115]{sr}.

  The first companies to use teamworking
  as the main method of production were Volvo, Toyota and
  General Foods in the early 1980s,
  and it attracted a lot of attention because no one
  else was doing it at the time. Since then, teamworking has 
  rapidly been expanding as a way of organising work and is now a well
  established in the world of management and most organisations have
  employees working in a team of some kind~\cite[284]{sr}.

  In the world human resources management, the term 
  ``teamworking'' is one of the hottest phrases used these days.
  One may argue that it
  has been promoted as the solution to all management problems and
  have conquered the industrial world since its introduction in the
  early 1980s~\cite[623]{ahdb}.

  In spite of all the positive publicity of teamworking,
  there is quite a lot of scepticism towards it. This is
  well illustrated by Hayes who
  states:~\emph{``team is a word for managers''}~\cite[374]{ahdb}.
  And to some extent he is right. It is not as simple as
  introducing teamworking in the workplace and expect increased production
  and greater job satisfaction. I will in this essay
  present the different conditions that
  have to be present in order to get successful teamworking.
  
  \section{Conditions for successful teamworking}
  \subsection{The team members }
  The perhaps most obvious condition for successful teamworking
  is the members of the team. Often this point is forgotten, but
  it is nevertheless extremely important to the understanding
  of creating good teamworking in an organisation.

  Teamwork requires its members to be comfortable with
  challenging tasks involving multiple skills and communication
  across traditional established networks. If the members of the
  team are reluctant to change, and to learn new technology
  and methods of working, the teamwork is deemed to fail.
  The members must value a continual learning process and
  must enjoy complex tasks~\cite[303]{sr}.

  \subsection{Task}
  The success of teamworking greatly depends on the kind
  of task the team is set to solve. Hackman (1987) claims that
  the task the team is given must include a need for a variety
  of high level skills~\cite[375]{ahdb}. The need for multiple
  skills for successive teamworking is also supported by
  Robbins (1998).

  Not all tasks are suitable
  for teamwork, and even within teams there are different kinds of
  teams suitable for different tasks. Robbins (1998) differentiates
  between three types of teams, namely problem solving teams, self
  managing teams and cross functional teams~\cite[287]{sr}.

  The first type is the one found in quality circles, which
  is often used by management to give the employees a sense
  of involvement and empowerment~\cite[88-9]{jhbm}.
  Quality circles involves sharing of information and ideas, but
  the teams themselves are given no autonomy or authority to
  carry out work as they see best fit. This kind of team
  is referred to as advice team by Meuse~\& Futrell,
  and is by that giving a good description of what the
  team is about; getting advice from the workers on how
  to best conduct the work at the company~\cite[378-80]{ahdb}.

  Self managing teams is the second kind of team according
  to Robbins. These teams are given the authority and autonomy
  to detect, correct and implement changes as they see fit in
  the workplace~\cite[287]{sr}.

  Cross functional teams is the last type of team according to
  Robbins, and it takes its members from different parts of the
  organisation. This team therefore operates across department
  boundaries and cultures, although it is restricted to
  the horizontal dimension of the organisation chart. The
  cross functional teams do not expand over the vertical
  dimension, challenging the hierarchical structure of a
  company.

  \subsection{Team size}
  Another obvious factor in teamworking, is the size of the
  teams. Small teams are good for action taking tasks, such
  producing a computer program~\cite[275]{sr}.

  After Ford's adoption of teamworking as its main
  organising mechanism in the mid 1980s, they have published
  what they believe is the context for good teamworking. In
  their list of vital conditions for teamworking,
  small groups is the first crucial element for success
  success in teamworking~\cite[299]{sr}.

  Cross functional teams is considered a medium sized group,
  and is a popular form of teamworking. It uses
  employees for different parts of the organisation to
  accomplish a task. The employees are selected from different
  parts of the horizontal dimension of the organisation
  hierarchy, but not the vertical. These teams are seen as a dynamic
  way of organising work, even though they may be hard to manage
  because of their size. Therefore it is advised to consider
  dividing teams of that size into sub teams~\cite[290]{sr}.

  When it comes to the actual definition of a small group, 
  the world of group psychology has a common rule that the magic
  number for small groups is seven. However, Belbin (1981) argues
  that six is the ultimate number. He argues that companies of
  six employees seems to be perform better than seven man
  companies, even though the seven man companies have the
  numerical advantage~\cite[115-7]{rmb}.

  Bigger teams, on the other hand, are good for decision making
  and fact finding tasks~\cite[275]{sr}. Groups are considered
  to a good thing when the quality of the output is the major concern.
  This argument is also supported by Vroom and Yetton who list quality
  as one of the three aspects to consider in
  connection with teamworking~\cite[743]{ahdb}.

  \subsection{Resources}
  Although Belbin is most famous for team roles,
  he argued also for other requirements that had to be
  met for getting successful teamworking, and
  resources was one of these elements.
  Belbin claimed that employees must be given enough resources
  in order to be fully empowered. This includes access to the
  appropriate technology, the project the team is to work with
  must be given enough money to be carried out in a satisfying
  manner. The team must also have the right machinery available
  in order to conduct successful teamwork~\cite[263]{ahdb}.

  \subsection{Feedback}
  Sundstr\"{o}m et al. listed feedback as one of the crucial points
  of successful teamworking. The team needs to get continuous feedback
  on the work they are doing. How the output is perceived by
  the rest of the company and how they are functioning as a
  team~\cite[396]{ahdb}. This is also supported by Hackman (1987)
  who argues that a team needs regular feedback on how they are doing
  in order to become an effective team~\cite[375]{ahdb}.

  The importance of feedback to the team is also supported by the
  outcomes of the famous Hawthorne studies in the 1920s and 1930s in
  the US. Workers' performance were measured under different
  conditions, such as different lighting, working times and
  food supply. The theory they had, was that the better
  the conditions were, the better performance the company
  would get from the workers. However,
  when the studies had been running for some time and the researches
  made the conditions worse for the workers, the production output
  to their surprise continued to rise. Therefore, one of the
  conclusions after the Hawthorne studies was that workers
  perform better when getting regular feedback and attention by the
  organisation they work for~\cite[12-3]{rh}.
  
  \subsection{Autonomy}
  Autonomy was another of elements pointed out by Belbin. The group
  must be given a decent amount of autonomy in order to be
  reach its potential~\cite[263]{ahdb}. This is also supported by
  Hackman (1987) who argued that the team members must be equipped
  with enough autonomy to decide how work is to be
  conducted~\cite[375]{ahdb}.

  Autonomy is also held as a crucial context for teamworking
  by Sundstr\"{o}m et al.. They argued that
  the task itself must hold be a mission of a certain degree
  of clarity. A team will not function well, if the intended mission
  is ambiguous or without a visible goal with impact on other people.
  Hackman (1987) argue that it is crucial for teamworking
  that the output of the teamwork is visible
  and that this output has consequences for other people. This, he argues,
  is vital for the team to sustain their motivation and enthusiasm for
  continuing working in a team and not wanting to either leave
  the company or revolt to other, more traditional ways of working
  in the company~\cite[375-8]{ahdb}. 
  
  The environment in which the team members are to work within must
  also be made suitable for teamworking. The physical surrounding must
  encourage communication between the members and not hinder
  any processes that is vital for this way of working~\cite[375]{ahdb}.

  Sundstr\"{o}m et al. also mentions that the organisation culture must
  be such that teamworking is encouraged and supported. Without an
  organisational understanding of what teamworking is, and how it
  can prosper, it is hard for the teamwork to evolve as
  planned~\cite[396]{ahdb}. How the teamworking will work in a longer
  time perspective is very much dependent on how the teamwork is
  perceived by the rest of the organisation and how it integrates
  with the rest of the company's activities.

  The development of the team itself is also important. If the
  team just retain the status quo, the team members' enthusiasm will 
  turn into apathy and the team will get ineffective.
  Therefore, Sundstr\"{o}m et. al. stress that
  a continuous development of the team is in itself a critical
  part of gaining successful teams~\cite[398]{ahdb}. This can also be related
  to the intrinsic rewards of teamwork together with the individuals'
  need for self actualisation, thus to develop their skills and to gain higher
  status in the group, as described by Abraham Maslow~\cite[22]{am}.
  
  \subsection{Culture}
  The national culture is another aspect to take into consideration
  when evaluating the prosperity of teamworking. It is argued that
  in cultures that has a great degree of collectivism, teamworking
  has better conditions for succeeding than in very individualistic
  societies~\cite[294]{sr}. 

  Collectivism here is used as described
  by Gert Hofstede in his studies of the cultural
  differences in 40 countries around the world~\cite[148-177]{gh}.
  He identified the degree of individualism (as opposed to collectivism)
  was one of four main dimension one could measure cultural differences.
  As an example, United States of America is listed as a very
  individualistic society, whereas Pakistan is one of the most
  collectivistic cultures.

  The company culture is also a very important issue to consider
  when assessing teamworking. The company itself must provide
  a culture and organisational structure that allows the
  emergence of teamworking and encourages it in practise and
  not just in theory as is often done. Teamworking as other
  ways of empowering employees have often stagnated on its way
  to implementation in organisations~\cite[101]{rw}.
  
  \subsection{Composition of group}
  Sundstr\"{o}m et al. also stresses the need for the importance
  of the right composition of a group. The group needs to have
  the right balance of technical ability, knowledge and interpersonal
  skills in order to function optimally~\cite[69]{rb}.

  R. Meridith Belbin conducted significant research on
  the area of team roles and how
  to accomplish the perfect balanced team. Belbin identified nine
  team roles that all needed to be fulfilled in order for the group
  to function properly together. This does not mean that a group
  has to consist of nine members. Quite on the contrary, Belbin
  argues strongly that the groups of six is the sacred number for
  small groups~\cite[115-7]{rmb}. However, all of the nine team roles
  should be fulfilled, thus some team members may take more than
  one role contributing to the full group picture.

  The first team role is the ``chairman'', who is
  the person that decides how to best utilise the strengths of the group
  and leads the group in its progress. The ``company worker'' is the team
  role that implements ideas and theories efficiently and effectively.
  The ``completer - finisher'' is the person that has the drive to get the
  projects finished and is aware of subjects that need especially much
  attention. The ``functional role'' is the member of the group that
  has the technical in-depth competence to assert the technical
  challenging tasks the group is facing. ``Monitor - evaluator'' is the
  team member, that by evaluating ideas and analysing problems,
  makes sure that the team is able to make balanced decisions. The ``plant''
  is responsible for contributing new ideas and strategies to the team,
  with attention to major issues. The ``resource investigator'' is a role
  that researches external sources and establishes network outside the team
  that the team can use in order to complete its tasks. The ``shaper'' is the
  one that specifies the shape of the work of the group in accordance with
  were the group is going. The ``shaper'' directs the attention to the important
  objectives and priorities. The last of Belbin's roles is the ``team worker''.
  This role is concerned with the common good of the team, drawing upon the
  major opinions in the group. The last of Belbin's team roles is
  the ``team worker'' who works for improving
  communication within the group and to build up the team spirit~\cite{rmb}.
  
  Another advantage of having clearly defined roles in a team is that
  it contributes to the identity of the individuals in the
  group~\cite[72]{rb}.

  Group diversity is seen as an asset to teamworking
  in the context of international human resource management, contributing
  to giving a clear understanding of local differences in the different
  markets the company operates under and helping the company to
  ``think globally and act locally''~\cite[7-8]{mmgo}. However, it is
  argued that a certain group cohesiveness influences the team's
  effectiveness~\cite[275]{sr}. Global, multinational companies therefore
  tries to make the teams in the organisation be moderately geographical
  cohesive, selecting team members from similar cultures to work
  together. This also makes it also easier for the employees
  to have a family life and still work for a global company
  that expects them to travel between multiple offices in different
  countries~\cite[18-36]{mmgo}.

  On the other hand, critics such as Brown (2000) argue
  that group cohesiveness
  is not necessarily a good thing as most human resource
  management books claim. He also questions the source of group
  cohesion, claiming that it is performance that causes group
  cohesion and not the other way around, as most HRM books claim.
  Furthermore, Brown argues that cohesion based on commitment
  to the group task gives far greater team performance than cohesion
  based on interpersonal commitment to the group~\cite[53]{rb}.
  
  \section{Leadership}
  The need of good leadership has been acknowledged for a long time. Le Bon,
  a contemporary of Freud, stressed the importance of a group leader, and
  described the leader as a person with with strong faith in his
  ideas and goals and with a strong drive to lead the
  will of the group~\cite[81]{sf}. 

  A team leader is both responsible for achieving the goals set for the
  team and is for the social organisation of the group~\cite[392]{ahdb}.
  Critics argue that the team ideology oversimplifies the need of
  leadership, that a self managing team without a traditional leader
  is an utopia that will not work in practise~\cite[304]{sr}.
  If there is no clear leadership of the team, the effect might be
  that the team get paralysed~\cite[293-301]{sr}.

  Also, in a group there is two kinds of leadership; one task specialist
  and one social emotional specialist. Both are leaders in their own right.
  The task specialist is the ``formal'' leader with responsibility for
  the progress and drive of the group. However, as studies have shown,
  there is also a social emotional leader in the group that is responsible
  for the social drive of the group, contributing to a good atmosphere to the
  teamwork~\cite[69]{rb}.

  \section{Trust}
  Trust is an important factor to be considered in the relationship
  between the leader of a group and the group members, and so also
  in a work team. In order to have effective teamworking, there must
  be a high degree of mutual trust in the team. The leader must
  trust the team members to do what they are supposed to do, and the
  members must be sure that the leader is doing what he is supposed, and
  also that he keeps his promises to the group, about rewards, goals
  and deadlines. Also, the individual team members must trust each other
  to do what everyone is suspected to, that there is no social loafing
  where some members take advantage of others doing their job.
  In as successful team, all the members are responsible for the
  common good of the team and their individual work~\cite[293]{sr}.

  Social loafing was as a problem of work was also recognised
  by Fredrick W. Taylor and was a problem that was addressed
  when he developed his theories of scientific
  management~\cite[98]{hb}.

  Strong trust does not come over night, and is built over time by
  a number of factors. A person needs to have certain integrity,
  that is a honesty and truthfulness over himself. Furthermore,
  competence is a vital source of trust and perhaps the most
  obvious one, since competence traditionally, is a primary condition
  for employment. The team members also needs to be consistent with
  what they are doing, by being reliable so that the others can
  rely on them delivering what they are supposed to the given time.
  Members must also have a good judgement in handling
  situations, seeing what needs doing at the given time. The team spirit
  also requires loyalty from its team members. Members must be ready
  so stand up for each other and defend each other from attacks from
  outside the group. The whole team philosophy naturally also claim
  a certain degree of openness from the team members to share
  information, ideas and rewards. Since teamworking is a collectivistic
  effort to solve problems~\cite[294]{sr}.

  \subsubsection{Rewards}
  Because teamworking is an act of collectivism, the reward system
  should also be collective. This way, the team members will feel
  that not only did they get intrinsic rewards by the accomplishment
  of the team output, but also the extrinsic rewards are
  in the spirit of collectivism, inspiring to further work in the
  team. This appreciation of Collectivistic rewards is also
  subject to the training of team members~\cite[293-7]{sr}.

  The importance of a reward system in successful teamworking
  is also supported Sundstr\"{o}m et al. who argue that both
  material rewards and intrinsic rewards such as recognition
  from the rest of the organisation is important factors
  in teamworking~\cite[396]{ahdb}.
  
  \section{Conclusion}
  Naturally, not everything about teamworking is of a positive nature.
  Failure in teamworking accounts for a staggering 70\%
  of all accidents in air traffic~\cite[374]{ahdb} and it is
  argued by several critics that the success teamworking is dependent
  on the task and teamworking is not applicable to all kinds of tasks. 

  Nevertheless, the popularity and
  expansion of teamworking has become substantial and continue
  to spread further into the world of management. One of the
  reasons for this, in addition to increased job satisfaction,
  higher quality solutions and creativity, is that teamworking
  gives additional benefits that otherwise would be missed by
  the team members.

  Employees get improved communication skills,
  expanded job skills and greater understanding of technology. Team
  members also get skilled in decision making processes and
  develop their interpersonal skills. This in effect is
  beneficial for the company that get more organisational flexibility
  with more people in the organisation that know more, about
  the organisation they work within and what other employees are doing
  in the company~\cite[303]{sr}.

  Sadly enough, teamworking is not the definite answer to all
  managerial problems. It is not a magic formula as many HRM books
  tend to argue, but it does represent a good, and sometimes
  superior, alternative on how to solve a great percentage
  of tasks companies face in the modern world of work.

  \newpage
  \setlength{\baselineskip}{1.3em}
  \bibliography{essay}
  \bibliographystyle{jmr}
  \nocite{*}

\end{document}

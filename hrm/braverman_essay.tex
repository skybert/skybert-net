\documentclass[a4paper,12pt,titlepage]{article}

\begin{titlepage}
  \title{Industrial Democracy \\and \\Employee Involvement \\ -- \\Essay Question 3}
  \author{Student Number 1121320}
  \date{November 29, 2002}
\end{titlepage}

\pagenumbering{Roman}
\pagestyle{plain} % plain is default
\usepackage{natbib}
\bibpunct{(}{)}{;}{a}{,}{,}
%\usepackage{harvard}

\begin{document}
  \setlength{\baselineskip}{2em}
  \maketitle
  
  \section{Introduction}
  Harry Braverman published in 1974 his book
  ``Labor and Monopoly Capital'' putting the focus
  on the labour process from the view of the worker.
  A viewpoint which is usually forgotten in the discussion
  of the work process~\cite[9]{mnpb}. Braverman work builds on
  Marx' economic theories on capitalism when describing how
  the process of work has developed in the modern industrialised society.
  One thing that distinguishes Braverman's work from Marx', is that
  Marx never lived to experience the industrial revolution
  and how his theories worked in this new world of industry.
  So even though their works cannot be compared, 
  Braverman does give new life to Marx' theories set in the modern world.

  One of the unique things about the
  Braverman thesis is that Braverman himself never has defended any of
  its critics or contributed to the debate in any way.
  This has its tragic explanation in that
  Harry Braverman died only two years after the publication
  of his book. Therefore, Braverman never lived to see the
  impact his book has made on the discussion of the labour
  process in the western world. One might also argue that
  this has created the ``Bravermania''~\cite[40-41]{ptcs}.

  The significance of Braverman's book can easily be measured
  by looking at the vast number of contributions to the
  discussion of his works. The discussion is quite amazingly
  still ongoing at the time of writing, 28 years after its
  publication. In 1998 there was even a whole conference
  dedicated to the discussion of ``Labor and Monopoly Capital''.
  Staged in Binghamton New York, it attracted scholars from a
  number of countries contributing to the debate. 

  Some scholars argue the Braverman's thesis is not applicable
  to the modern work process, that it has been proved to be invalid
  by the sum of its critics. Some critics argue that the thesis
  must be corrected, whereas other critics go further and claim
  that the whole thesis have been contradicted and proved invalid.
  One of these critics, John Eldridge, puts it like this:
  \begin{quote}
    "Braverman's thesis has died the death of a thousand qualifications"
    John Eldridge
  \end{quote}
  
  This essay will outline Braverman's thesis, how it came about
  and most importantly, what it is about. Secondly, the essay
  will present the different critiques to Braverman's thesis,
  the background for these and whether they are to be considered
  valid. Finally, I will assess if and to what extent, Braverman's
  thesis is applicable to real world development in the world of work.
  
  \section{Braverman's Thesis}
  Braverman spends a lot of time in his book addressing Frederick Winslow Taylor's
  principles of scientific management and therefore I find it logical
  to start here when outlining Braverman's thesis.

  One of Taylor's themes that Braverman focused on was the claim that
  as long as the workers hold the knowledge of how to do the work, and that
  the worker's sum of knowledge exceeds management's knowledge, it will be
  impossible for management to control the production output. This is because,
  Taylor argues, the workers will use their power in knowledge to
  control the management's influence. If the management is behaving
  unreasonable, according to the workers, for example by requesting
  increased output, the workers will just hold back, since they know that
  the management are dependent on them~\cite[100]{hb}.

  Taylor therefore started to make the workers replaceable by
  dividing work into easy manageable tasks. So that the different workers
  carrying out the different tasks first of all did not need to know the
  other tasks, and secondly, the skill required for the different tasks
  was a lot lower than the required by the previous all around worker.
  This way, it became easier for management to simply  replace the
  workers they were not satisfied with, since the requirements for
  the different jobs were a lot lower and the time spent to train new
  workers was considerably lower.

  Another important part of Taylor's theories is that the conception
  of work is separated from the execution of work~\cite[151]{mnpb}. All
  the brain work of the job, all design and planning, is carried out in
  the office. The worker on the shopfloor should not need to think about
  the work, the worker should use all his or her energy to carry out the
  carefully designed task~\cite[113-120]{hb}.

  According to Taylor this had also a positive impact on the worker.
  Since the work was separated in to small easy manageable tasks, the
  worker could perform way more advanced tasks than he ever could
  have done without the scientific division of labour. Taylor argued
  that with his division of labour, low skilled workers could
  do work that before only belonged to workers with far greater skill,
  impossible for the normal worker to participate in. Quite the opposite
  of what all critics accuse Taylorism for, Taylor himself described
  this as a move towards ``more brain and less monotony''~\cite[129]{hb}.

  This deskilling of labour had more reasons that management got control
  over the total development process. This division and management of
  labour also meant higher efficiency because the workers all used
  the most effective, ``best way'' of doing their tasks. It was also
  beneficial for management because it made the production more
  predictable. The use of scientific management gave the management
  a greater guarantee for when work was finished and how much outcome it
  resulted in~\cite[434]{ahdb}.

  The core of Braverman's thesis is that management seeks
  to control the entire labour process in order to make sure
  that it gets the most out of the worker and thereby secures
  a maximum output of the production. Braverman talks about
  volarisation, that there are two values that makes up the total
  value of the workplace. One is the value of the work once in
  development and one once the work is ready for sale. In order
  to achieve this, management needs to control the entire
  labour process, the pace of work, how work is rewarded and maybe
  most importantly, how work is carried out in order to
  gain the maximum output from production~\cite[151]{mnpb}.

  Braverman argues that there are two kinds of deskilling taking
  place in the work place,
  one technological and one organisational. The technological is the
  idea that work shall be divided into easy doable tasks, where the
  skill is left to the technology rather than the worker. He argued that
  the separation of brain and hand is the work of capitalism. The brain
  work should be left to the office, not for the worker, following
  Frederick Taylor's second principle of scientific management.
  Therefore, Braverman also talks about a organisational deskilling~\cite[113]{hb}.
  He also found that this deskilling was recursive, the
  technological and organisational deskilling took place all
  over again in the office as it before was done
  on the shopfloor~\cite[316]{hb}.

  There is also a fundamental conflict of interest between the worker
  and the capitalism according to Braverman. This is because meaning
  of work, at least under capitalism, is to create a profit, leaving
  no room for the worker's satisfaction~\cite[13]{sw}. Capitalism
  also have a monopoly of the means of production and the whole
  capital backup up the industry. The worker is only a subject
  to capitalism, selling his or her 
  ``power to labour over an agreed period of time''~\cite[54]{hb}
  and thereby is cut off from having shared concern or responsibility
  for the work place, other than having a place to work fulfilling
  the need to work and the money this rewards the worker.

  \section{Critics}
  Harry Braverman's book have been very important to the debate
  about the nature of work,~\cite{sw}. It has also been subject to
  a lot of criticism and articles are still published to contribute
  to the debate. \cite{mnpb} have categorised both the critics and their
  criticisms into groups. They split the debate contributors into two
  categories, sympathisers and agnostics. The sympathisers are the critics
  that accept the general concepts of the Braverman thesis, but offer some
  minor refinements to his work. Agnostics is the other group of critics. This
  group do respect some of the elements of Braverman's work, but generally
  find his work out of aim~\cite[153]{mnpb}.

  \subsection{Workers' Resistance}
  The number of criticisms to Braverman thesis have been grouped into
  six different groups. One of these criticisms is that Braverman
  treats the workers as passive,
  that it is given that workers have total consent towards management's decisions
  and do not oppose them~\cite[12]{sw}. This opposition could either be individual,
  collective or powered by a trade union~\cite[435]{ahdb}.
  This is supported by~\cite{rp} that
  points out that Braverman does not leave any room for how social relations
  can work in the work process. Braverman believes that political struggles
  in all ways are separate to the work process~\cite[43]{rp}. 

  This point is also supported by~\cite{re}, that claimed that Braverman
  ignored worker's resistance and backed the argument up with research
  showing how management had to rethink their management
  strategies and how to control the workers. Even
  though management managed to defeat the worker's resistance to change of the labour
  process, the workers' resistance were of a such considerable character that
  management felt forced to find more advanced ways of
  controlling the workers~\cite[48-71]{re}.

  Knights and Willmott (1990) claimed that the greatest hole in
  Braverman's work was the absence of the worker's resistance. They described
  this as the active element of the labour process,
  the ``living individual subject''~\cite[53]{ptcs}. Edwards and Friedman are
  referenced in~\cite[16]{sw} giving examples of how management needed
  to develop new ways of controlling labour because of workers' opposition.
  This argument is also supported by~\cite{re} who argues that at least
  in the twentieth century, opposition from workers has become so frequent,
  that management had to reorganise the whole process of labour. This
  had to be done not just for reducing the workers' ability to resistance, but
  also to alter the workers' perception of the motives of opposition~\cite[16]{re}. 

  This idea is also supported by Armstrong (1989 and 1995) cited
  in~\cite[155]{mnpb}. Armstrong claimed that the influence represented
  by management accountances at board level is so significant that
  management is bound to rethink their strategy to be in lines
  with financial objectives as well as human resource objectives.

  \subsection{Ignores Gender}
  A third criticism of Braverman is that he totally ignores the
  aspects of gender in his deskilling thesis.
  This leads to a number of problems when trying to get the total
  understand the sexual division of labour~\cite[54]{vb}. One of the
  things Braverman is accused for in this respect, is that he totally separates the
  family from the labour process. He makes no reference to the family
  when discussing the labour process and when he discusses the family it
  is totally isolated from the labour process. Therefore
  ``Labor and Monopoly Capital'' does not provide a foundation for
  understanding the development in the sexual division of labour
  and the traditional, patriarcical social relations that exists
  in the capitalist organisation~\cite[71]{vb}.

  Other critics such as Baxandall, Ewen and Gorden (1976) cited
  in~\cite{vb} criticise Braverman of not assessing how monopoly capitalism
  has affected the role of the housewife and that he has failed to
  examine the connection between the social division of labour
  and the detail division of labour. Furthermore, they argue that
  Braverman has ignored women's unpaid labour in the home and to
  acknowledge the housewifery as craftsmanship. Braverman also
  fails to appreciate women's work on the domestic arena.
  This craftsmanship, they claim,
  has also under the influence of capitalism been deskilled and the
  domestic arena has been turned into an ``internal market'' for
  consumption within monopoly capitalism~\cite[54]{vb}.

  \subsection{Ignores Skill Transfer Possibilities}
  \cite{rp:smwlp} is one of the contributors to the debate
  that represents another critique of Braverman. The core of
  this critique is that Braverman ignores the possibilities of skill
  transfer in the labour process. Deskilling in one area of work
  may result in upskilling in another area. Penn
  uses research from the textile industries in Rochdale Lancashire
  as an example
  and conclude that there was a tendency that management issued
  a deskilling in production in order to gain more control over the
  process and increase the predictability of the outcome, according to
  Braverman's thesis. Where Penn's research differs from Braverman however,
  is that he discovered there was an increase in the need for
  higher skilled labour within organised labour. He claimed that
  it was both through pressure form strong trade unions and the
  fundamental need for higher skill in some areas in the industry,
  that lead to an increase of higher skilled work~\cite[107]{rp:smwlp}.

  \cite{sw} is another critic that claim that deskilling in some
  areas of the workplace may introduce upskilling in other areas. 
  Wood claims that even though deskilling
  may occur some parts of the industry due to technology, the general
  tendency in modern industry, is that technology leads to a number
  of new skilled jobs, especially in the electronics industry.~\cite{sw}
  also argue that Braverman fails to acknowledge that workers in
  the nineteenth century lacked the basic skill of literacy. These
  days this skill is implied for almost all kinds of work and
  not recognised by Braverman as an upskilling of labour when
  looking at the developments of the working class~\cite[18-19]{sw}.

  Together with other critics such as Scattergood, Gasteen and Sewel, Penn
  argues that neither deskilling or the opposite, upskilling, is by
  themselves enough to understand the development of the skill change
  in the labour process. They claim that there is a compensationary
  theory, that technological change leads to both deskilling and upskilling.
  Zuboff (1988) analysed this in detail and found that with the
  continuing emphasis on technology in the workplace led to a decline
  in action centred jobs, the jobs where the workers physically
  do something, and an increase in the need of intellective skills, that is jobs
  involving analysis. These ideas are called the polarisation of
  skills~\cite[161-165]{mnpb}.
  
  \subsection{Worker's Consent}
  Another great critic of Braverman is that he underestimates the workers'
  consent and accommodation by employees~\cite[155]{mnpb}. This implies the
  workers do not necessarily have a problem with their own subordination to
  management and will not want to try challenging this authority
  relationship. This idea was first put forth by Buraway (1979) that argued
  that workers instead of trying to challenge management's way of controlling
  the labour process, the workers themselves developed an internal, informal
  work culture that offers other ways of looking at the work situation.
  Thereby giving the workers more meaningful activities, since they look
  upon the work situation differently.

  The degree of workers' consent to their own subordination can also be
  found in the works of Karl Marx. \cite{mnpb} presents Marx' view on
  the worker selling his labour as follows:
  \begin{quote}
    ``\ldots in selling their labour power employees are relinquishing the
    right to control their own labour, thus discretion over how
    and when work should be undertaken becomes the
    prerogative of employers'' Marx cited,~\cite[229]{mnpb}
  \end{quote}
  Marx argued that it is implied when workers sell  
  ``the power to labour over an agreed period of time'' as
  Braverman defines the selling of labour~\cite[54]{hb},
  that the management can decide how and when work is to
  be carried out. The worker should be fine with this since
  they have given their consent by selling their labour power.
  Their reward for this is the wage paid by the employer.

  \subsection{Overstates Management's Objective of Controlling Labour}
  Braverman is criticised for overstating the Management's objective
  of controlling labour. Critics such as Kelly (1985) argue that it
  is a lot more to it than just controlling labour in order to
  secure a profit. Controlling the labour process only belongs to
  one half of volarisation, ignoring the second half, namely
  securing a profit through sale of the commodities on the markets.

  Also, Braverman is criticised of taking every change made
  by management to be in the interest of gaining more control
  over the work process. The introduction of new technology is such
  an example. Technological deskilling
  was to transfer power and autonomy from the shopfloor to the
  office~\cite[151]{hb}. He argued that the development
  in technology only broadened the gulf
  between workers and the machine~\cite[230-232]{hb}. However, as
  critics such as ~\cite{bglm} argue, management may have many other
  reasons for introducing new technology, other than increased control
  over the labour process~\cite[206-207]{bglm}.

  \cite{rcsr} supports this with giving an example of advantages
  of the introduction of the computer in the clerical industry. The
  computer put less stress on the individual clerks and their
  experience to validate the data on their respective cases. A computer
  could be used for validating the accuracy of the data entered, even
  though this gives management more control over what the clerk is doing,
  at the same time, distributed computing gives the clerk that chance
  to follow the data through all parts of its processing. The computer
  can also store all cases the clerk has been working on leaving
  greater overview for the individual clerk~\cite[177-179]{rcsr}.
  
  Technology may give management new opportunities in
  terms of product quality, increased product development, efficiency and
  flexibility~\cite[155]{mnpb}. New technology could make the
  company better equipped to meet fluctuating market demands, being
  more able to broaden its appeal if the market demanded it.

  \subsection{Ignores Alternative Management Strategies}
  Friedman (1977 and 1990) cited in~\cite{mnpb} is one of the critics
  arguing that Braverman ignores the fact that management may have other
  strategies to meet their goals. Friedman had especially new group
  technologies such as job enrichment and quality circles in mind. They all
  represent new ideas wants to make the workers more autonomous and
  thereby take more responsibility.
  
  One of the most famous examples of quality circles are the
  Volvo and Saab car plants in the 1970s. These were social groups
  that worked together for accomplishing a certain task. They
  had a certain degree of autonomy and responsibility making them
  very different to the traditional, assembly line model of
  car production~\cite[32-65]{mcc}. Quality circles are an
  ``advice team'' according to Eric Sundstrom,
  Kenneth De Meuse and David Futrell and one of the primary
  tasks of ``advice teams'',
  except for carrying out their work, is to provide management
  with feedback and advice upon production related issues. This
  way, there is a two way communication in the organisational
  hierarchy~\cite[380-381]{ahdb}. The workers feel empowered
  and contribute in a more flexible way to the company. 

  Another aspect of this is that by giving some discretion to the workers
  in an assembly line setting, the workers become more interchangeable,
  moving more easily around the line since they themselves take
  responsibility for the work. This in the next turn leads to greater
  assembly line balancing and greater work satisfaction for the workers.
  At the same time, management continues to control the labour
  process~\cite[435]{ahdb}.

  \cite{sw} argues that one of the main problems with Braverman is
  that it takes for granted that the logic of Taylor is the logic of
  capitalism. That is, the objective to control the labour process.
  Braverman is criticised for having no historical proof of this,
  that he implies that before Taylorism, control was not in the
  hands of management, that management were given control only
  for practical reasons. The real power was still in the hands of
  the workers since their knowledge superseded management's
  knowledge~\cite[15]{sw}. This point is supported by Taylor himself,
  that wrote that as long as the sum of all the workers' knowledge exceed
  management's knowledge of work, management could not control
  the labour process~\cite[100]{hb}.

  Braverman is also criticised for not being accurate enough
  in his historic background of the nineteenth
  century.~\cite{cl} argues that Braverman oversimplifies how the
  control of labour was in the nineteenth century industry. Littler
  emphasise the difference in the structures of control in Britain
  and how it is presented in Braverman's work. Furthermore, Littler
  stresses that the contract system had great control over the
  labour process and that Braverman is romanticising the craft
  control by the worker in the nineteenth century
  industry~\cite[122-145]{cl}.

  \section{Supporters}
  Some defenders of Braverman's thesis, such as Armstrong (1988),
  claim that Braverman tried to describe a general ``law of motion''
  in capitalist economies. This could be influence in either way by
  a number of factors, which have been more or less discovered by
  Braverman's critics. In order words, what the critics of Braverman
  suggest of modifications of his thesis, could be seen as just realisation
  of such changing factors. Braverman tried only to describe a general
  tendency in how the work process is being shaped in a capitalist
  economy~\cite[156]{mnpb}.

  Braverman also gets support from~\cite{ccpt}, that concludes in their
  report on the change in the accounting craft that there is still
  truth to Braverman's thesis in today's industry.
  Cooper~\& Taylor looked at advertisements for accountants and
  how the requirements for these positions changed from 1974 to 1996.
  What they found was that even though there was a significant higher
  demand for computer experience the further out in the research period
  they looked at, the range of other qualifications radically declined.
  What was once a profession that demanded a number of skills, was
  reduced to a few critical skills, with a heavy emphasise on computer
  skills. This implied a number of issues. First, it seemed like
  a lot of accounting work was degraded to simply computer
  punching of data and the real brain work was left to a few, highly
  skilled accountants, leaving out a big number of previously
  low and semi skilled accountants. Secondly, the tendency found
  was that accounting was looked upon as something more and more
  firms outsourced to professional accountant firms, leading to
  fewer accountants jobs on the whole.

  \cite{ccpt} argue that
  Braverman not just represents a significant theoretical basis in order to
  explain the development in the requirements for the accounting
  craftsmanship from 1974 to 1996, but his analysis of Frederick Taylor's
  scientific management will also continue to be influential in years to come
  the way the accounting craftsmanship is going~\cite[575]{ccpt}.

  A lot of the same conclusion drawn by~\cite{ccpt} can be found
  in~\cite{rcsr} and their research of the change in clerical work.
  They do recognise the deskilling of the clerical position, that
  the clerical position once required detailed knowledge and had
  a significant degree of autonomy. Technology has been introduced
  in the clerical profession with computers and the clerk has become
  more a subject to the computer and have a lot less
  responsibility for coordination and completion of separate work tasks.
  These tasks are left for management, to those who plan 
  and coordinate the use of the computers~\cite[175-78]{rcsr}.
  
  \section{Braverman in the Modern World of Work}
  According to Braverman, all new organisation theories are
  based on Taylorism~\cite[87]{hb}. However, as Braverman also writes
  the degree varies on how much of Taylorism the different organisations
  have inherited. An interesting note is that Braverman mentions
  farmers as a unique group of workers
  that fall outside much of his theories of how everyone is an employee
  of capitalism~\cite[109]{hb}. I will now look at some examples
  illustrating the validity of Braverman's thesis in
  the development in the world of work. 

  McDonald's is perhaps the best example of Taylorism in the modern world of
  work. The multinational corporation has restaurants in all corners of the world
  and is continually expanding. Braverman deskilling has since the 1990s been
  discussed as the ``McDonaldisation'' thesis~\cite[433]{ahdb}. And there
  may be a lot of true in that the ``McDonald way'' has incorporated Taylor's 
  scientific management throughout its organisation. Following the Taylor
  theories, they have deskilled all parts of the process of running a
  hamburger restaurant. The recipes for each of the burgers are
  the same no matter which McDonald's restaurant you choose, a ``Big Mac'' is
  a ``Big Mac'', regardless of purchasing it in Manhattan or Melbourne.
  The French fries come pre cut in plastic bags to the restaurant, same
  for vegetables and sauces. There are alarms set for the different boiling
  and frying times and the employees even have standard ways of greeting
  the customer. Everything is done in order to make the ``McDonald's experience''
  the same no matter where you are in the world. It is both a matter of
  branding, but maybe even more importantly, a case of management taking
  total control over the labour process~\cite[143-144]{mnpb}.

  Call centres is an increasing industry as more and more firms rely on them.
  The call centres customers include everything from emergency calls
  to support and customer services. It has become a trend that companies
  out source this kind of operation to professional call centres, instead
  of handling it themself, as would be the traditional approach.

  If we examine the call centres' mode of operation we will also
  find that in itself it represents an example of a deskilled workplace.
  This may sound a little surprising when first thinking about it since
  a call centre operator's job is about having conversations with people.
  Human interaction should in theory be something unique and different
  every time the operator gets a call, however this is not the case. The
  call centres operates with set schemes and rules for what their
  employees are to say in all phases of the conversation with the
  caller. Everything from the opening greeting to the problem solving
  and closing remarks are carefully described by the call centre management,
  leaving the individual call centre operator little choice and autonomy
  in their daily work~\cite[58]{ptcs} and Taylor~\& Bain (1999) cited
  in ~\cite[436]{ahdb}.

  Call centres even take the time of how much time the operators
  use on the different stages in the conversation, keeping statistics
  with resolution of one hundredth of a second.
  The call centre operators are divided into
  teams that are ranked on a weekly basis on the average call
  handling time of the different team members. 
  
  We can also see a sign of Marx' and Braverman's theories in that
  the worker becomes alienated towards its work because the worker does not
  have a direct relation to what he or she is doing. Doing support on a
  product that the worker's firm have not made is a source of alienation.
  This fact makes a distance from the worker to the product, 
  since the call operator is not an employee of the firm delivering
  the service or product the caller is enquiring about.

  Another example of an industry that implements Taylor's principles
  to a great extent, and in doing so creates an alienated work situation,
  is the chicken industry. 

  \cite{mnpb} prints an interview with an assembly line working on
  a chicken plant, done for a television programme called ``Dangerous Lives''.
  The worker's only job was to check chickens for livers, hearts or
  anything else that might be inside the chicken. Then to take everything
  out into bins on the floor. This work was performed at an
  incredible pace counting 2000 chickens an hour, 14000 chickens a day
  leaving the worker highly alienated to his work~\cite[147-148]{mnpb}.
  
  The most famous example of Taylorism is probably Henry T. Ford's
  car plants. Ford revolutionialised the car industry in the early 1900s.
  Nothing he did was totally new, he did not make a single new
  technological invention. What he did was merely that he used already
  developed technologies and put them together, with a clear idea
  of what he was to achieve and how to do this.
  The first thing he did was the to use public relations to promote his
  cars. He was the first in the industry to do this to a great extent.
  Secondly, and for what he is most famous for, he introduced the
  assembly line in the car plant. Instead of making the workers move
  to where the work was, he made the work come to the workers~\cite[18]{be}.

  Henry Ford had the most famous early implementation of Taylorism in
  the industry. By following the theories of scientific management, he
  divided work into easy doable tasks, had performance related pay and
  kept the workers under a strict supervision~\cite[146-148]{mnpb}.

  Braverman explicitly described the division of labour at Ford's car
  plants and described it as a fine example of capitalistic control over
  the labour process and how they followed Tayloristic
  ideas~\cite[146-151]{hb}.
  
  Checkout operators is another work group that really have become
  a regular sight in stores the last 30 years, especially in
  larger towns or cities. The normal operation of work for
  a checkout operator is quite different these days to what it used
  to be. If we go back to the time of 
  ``Labor and Monopoly Capital'', checkout operators had to
  know the price of the individual commodities and punch this
  in on a calculator to calculate the final sum. Some places were
  even still using pen and paper (some still do today but it is rare).
  The last 15 years it has become more common that a machine
  registers the price of the different goods, taking the
  brainwork from the worker and putting it on the computer.

  The checkout operator does not even have to make out
  how much money the customer is to get back in change, since
  the computer also calculates this for the checkout operator.
  The checkout operator in this respect has been deskilled in
  that he or she does not need to hold the same amount of
  skills as before the introduction of technology. This deskilling
  follows exactly what Braverman writes about capitalism
  wanting to separate the brain and work in order to reduce
  production time, that is the time to calculate the sum of
  the prices of the customer's goods in this
  connection~\cite[120-126]{hb}.
  The only real skill the operator still needs to hold, is social skills,
  giving the service the customer expects.

  \section{Conclusion}
  I have now looked at a lot of the critics that have arisen after
  the Braverman's thesis was published alongside with defences and
  real world examples of Braverman's theories in the modern world
  of work.
  
  Braverman emphasised in his introduction that his work
  was not meant as a nostalgic cry on the lost era of
  craftsmanship, but merely as a comment on the contemporary
  development of capitalism in the US industry and how it
  affected the cycle of work. He stated that he did not want a 
  crusade to old times of labour, but merely encouraged
  a way that labour and technology could work
  together in the future~\cite[7]{hb}.

  This can be important to bear in mind when reading his works,
  since it is a quite dramatic description how everything has
  become an employee to capitalism and how management and capitalism
  have arranged work in order to secure profit and development.

  One of Braverman's greatest defenders, Armstrong, argue that
  a lot of the critics' corrections and ideas to the Braverman's thesis
  is already there in the original text, one only have to have the right
  eyes to ``see'' them. Rereading the original text would
  very often reveal the same ideas put forward by the critics.
  He also accuse the critics of using too narrow scope in
  case studies to prove Braverman wrong,
  as Braverman described general tendencies in the industry~\cite[45]{ptcs}.

  And this is true, a lot of Braverman's critics
  can be defended by rereading the ``Labor and Monopoly Capital''.
  For example, when one of the most common
  critics to Braverman's thesis, that he did not take into account the worker's
  resistance to change, is described by Braverman himself,
  He admits already in the introduction of his book that he did not
  include the dimension of the modern working class and
  their degree of self awareness, organisation and activity. And
  in that lies all the individual worker's and the collective
  resistance to management's introduction of scientific management,
  in addition to trade unionism's opposition to capitalism's control
  over the labour process~\cite[26-27]{hb}.

  Regardless of the critics, there is a couple
  of issues that are important to remember regarding the Braverman
  thesis. First of all it was a description of general tendencies
  of the development in the labour process in the United States.
  Secondly, the cultural differences, both work wise and socially between
  the UK and the US is so different that this dimension must
  be taken into account when reading and assessing Braverman's text.
  Secondly, the Braverman book is a comment on a \emph{general}
  tendency, not a rule book but a general ``law of motion''
  in capitalist economies, Armstrong cited~\cite[156]{mnpb}.
  And I believe Armstrong is right. Braverman must not be
  read like an instruction book, but rather represent a
  valuable discussion on where the labour process development
  is going in a capitalist economy. As shown in the different
  real world examples such as McDonald's and call centres,
  Braverman's thesis has by no means
  ``died the death of a thousand qualifications''.
  
  % toc and bibliography
  %  \newpage
  %  \tableofcontents

  \newpage
  \setlength{\baselineskip}{1.3em}
  \bibliography{essay}
  \bibliographystyle{plainnat}
  \nocite{*}

\end{document}
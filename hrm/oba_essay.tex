\documentclass[a4paper,12pt,titlepage]{article}

\begin{titlepage}
  \title{Organisational Behaviour and Analysis \\ --- \\Essay Question 3}
  \author{Student number 1121320}
  \date{November 11 2002}
\end{titlepage}

\pagenumbering{Roman}
\pagestyle{plain} % plain is default
\usepackage{natbib}
\bibpunct{(}{)}{;}{a}{,}{,}

\begin{document}
  \setlength{\baselineskip}{1.7em}
  \maketitle
  
  \section{Introduction}

  Ever since large groups of people started working together
  for accomplishing a certain task, the ones in charge 
  have been concerned with how to make the workers
  perform their best in order to get the work done as quickly and
  good as possible. Back in history, most work of scale
  were performed by slaves, and the ways of making them perform
  their best were highly questionable~\cite[15-38]{a}. However, if we look
  towards work organisations after the demise of slavery, there
  has been a development and change in how work groups are
  organised and this is what will be addressed in this
  essay.

  The later years, it has been a significant shift in the industry
  towards higher job satisfaction for the workers. The workers
  is in many ways a lot more demanding these days, as they want more
  from the workplace than just the monthly pay check.
  Noon~\& Blyton writes:
  \begin{quote}
    ``For most people work is a place to socialise,
    and complex social systems develop within the
    workplace which often spill over into leisure time.'' \\
    \cite[77]{nb}
  \end{quote}

  Many companies recognise this fact that work for the worker is also about
  social relationships and a sense of belonging. The social aspect
  of work is also important to the employer as it has an impact
  on how the workers perform. Howtorne claims that it is psychological
  rather than physical factors that influence the effectiveness on how
  we work~\cite[32-65]{mcc}. 

  Therefore, it has become more and more common for firms to
  encourage social activities among the workers both in and outside
  the workplace. Optimally, these activities should include workers
  from all departments and levels of the company hierarchy in order to
  break down the barriers between the physical boundaries between
  for example the marketing department and the technicians in
  a nuclear power plant. By having a social relationship to the
  fellow workers, workers will function better as a ``team'' at
  the workplace, solving problems across hierarchical directions
  easing the pressure on the management and increases the feel
  of involvement and identity with the firm.

  \section{Traditional Organisation Design}

  Modern industry as we know it,
  saw a radical shift in how groups were organised with
  Henry T. Ford's introduction of mass production
  of cars in 1903~\cite[426]{hb}.
  
  Prior to Ford, goods had mainly been manufactured
  in a the way that the worker followed the product from its beginning
  to its end, thus were involved with more or less the whole production
  cycle. Inspired by Taylor's theories of scientific management~\cite{vd},
  Ford introduced quite a few new ways of producing cars.
  The most famous element was the assembly line and the idea that
  each worker did a small, clearly defined task, contributing a small
  piece to the big jigsaw puzzle of producing a car.

  This way of organising the workforce, had obvious advantages
  according to Taylor and Ford. First of all, it lowered the
  entry requirements for the workers. Almost everyone could join the
  assembly line, since the tasks were very simple and all the
  thinking were left to the technology and the management, thus
  totally separated from the workshop employee. Secondly, it made it easy for
  management to control the production cycle and predict the outcome
  of production. Third, the assembly line workers became also easy to
  replace since they did not hold any particular skills.

  As Huczynski and Buchanan (2001) argues,
  we can find examples of Taylor's theories and Ford's practices
  in many parts of the world's industry today~\cite[433-435]{hb}.
  However, there are several limitations and critics to the Fordist way,
  and thus other ways of organising groups have emerged. 

  \section{Working in Teams}

  One of the fundamental problems of Taylorism, is the pure nature of the
  way of work. According to Taylor, the only motivation for the worker
  was the money~\cite[32-65]{mcc},
  in order words to full fill the fundamental needs as
  described in Maslow's hierarchy of needs~\cite[240-243]{hb}.
  However, as many theories of motivation
  suggests, the human being is driven by more things than just food,
  housing and material status, the so called extrinsic values.
  Other values are as important, namely elements such as job satisfaction,
  the sense of accomplishment, being able to identify oneself with the
  product one is making and the feeling of making a difference. These are
  a part of the intrinsic values~\cite[95]{rw}.

  These days, many managers talk of teamworking and there has been
  a significant shift towards organising workers in teams~\cite[375]{hb}.
  Critics, such as Womak,  argue that
  teamworking really is only a fancy buzz word used by management loring
  the workers to think that they get more autonomy and control, as it
  really only gives the management more control and may even take away
  power from the workers~\cite[393]{hb}. This is supported by
  Benders and Hootegem (1999) who claim that there is a lot of hype
  around the term ``teamworking'', that is is merely  a word managers
  use in order to achieve their goals without giving anything more.

  Nevertheless, teamworking has made significant development in
  today's industry. Three scientists, namely Eric Sundstrom,
  Kenneth De Meuse and David Futrell have identified four different kinds
  of teams, namely advice teams, action teams, project teams
  and production teams~\cite[378]{hb}.

  Teams are an attempt to make improve both work experience, 
  and by that improve the performance of the company.
  The basic ideas behind teamworking is to both improve job satisfaction,
  employee involvement and increase productivity. Also, since teams are
  given its own internal authority, they are to become more autonomous
  units that can manage smaller problems themselves such as
  supply of material and smaller changes in production without going
  the way of management, thus leaving less stress on management that
  can again use their time more effectively, without having to
  worry about ``minor details''.

  The first kind of teamwork identified by
  Sundstrom, Meuse and Futrell is ``advice teams''. The main idea here
  is that the teams shall function as advisers for the management,
  providing feedback for further decision making. At the same time,
  it is to give the workers a greater feeling of involvement and
  influence in their workplace. Quality circles is an example
  of advice teams~\cite[380-381]{hb}.

  The Swedish
  car manufactures Volvo and Saab were for many years shining examples
  of firms successfully implementing this way of organising the
  workforce~\cite{mcc}. The idea of quality circles is that workers in the
  workshop are organised in groups that several times a day, meet
  and discuss how to best run the production. This idea was introduced
  by Herzberg in 1966. In his study he looked at the reasons
  for job satisfaction and job dissatisfaction. He found that
  dissatisfaction was caused by what he called hygiene factors, which
  can be described as general welfare at the workplace. Job satisfaction
  on the other hand, was due motivators, that is rewards in one form
  or another. Quality circles was a way of giving the workers
  a greater feel of job satisfaction~\cite{mcc}.

  Toyota is another example of recent implementation of the quality circle
  idea. Also called ``Toyotaism'' is an example of Japanese teamworking,
  where several elements in how the industry is run are totally different
  to traditional industry in the western countries. Japanese teamworking
  also involves elements such as ``Total Quality Management'' (TQM), to
  improve the organisational operation towards perfection,
  and ``Just in Time'' (JIT) systems. That is, products are delivered
  as they are produced, hence the firm does not have to keep a stock of
  its products. The reason behind both techniques were
  of course  to improve company competitiveness.

  Ouchi (1981) described another element of Japanese management
  as ``Theory Z''. The concept is that although management gives
  a clear strategy and overall goal theory for the company, the
  actual implementation of this strategy only becomes clear as it
  finds its way downwards in the organisational hierarchy. This
  gives the employees greater independence and involvement in
  the production cycle~\cite[529]{rw}.

  Even though Japanese management, or Toyotaism,
  unarguably represents an example of
  teamworking in the manufacturing industry, critics argue
  that Japanese teamworking has more elements from Taylorism than of
  employee empowerment~\cite[421]{hb}.

  A different kind of teamworking is ``action teams''.
  These teams differ from advice teams in that they are
  comprised of highly skilled labour that are put together to do a
  specific task for a set time period.

  We can find a very good example an action team in a Formel 1 race.
  When the racers make a pit stop to change their wheels and get more fuel,
  13 workers, specialised only for this task, works intensively
  for 4-6 seconds to accomplish their task of giving the
  driver new wheels and petrol~\cite[381-384]{hb}.

  The third kind of teamworking according to Sundstr�m, Meuse and
  Futrell is ``project teams''. These teams are groups put together
  for a limited time perspective to accomplish a specific task. 
  This task is often referred to as a project, hence the
  name~\cite[386]{hb}.
  
  The workers are often put together from different parts of the
  organisation, something which gives a greater feel for unity within
  the organisation and contributes to break down barriers between
  the different groups in the organisation.

  However, the members of these groups are usually chosen from the
  same level in the organisational hierarchy, so the effect of
  barrier breaking is limited to the horizontal
  dimension of the organisational structure~\cite[384-387]{hb}.

  Production teams are the fourth and last kind of teamworking.
  These groups are teams that are responsible for the day
  to day operation of core operations in an organisation.
  For example can the group responsible for pane in a
  window plant be called a production team.

  This kind of teamworking is not at all new and can be found
  in most, if not all, big factories. A perfect example of
  production teams can be found at the Ford car plants.
  Even though everything about the production is fragmented,
  the factory itself is divided into different ``shops'', that
  represent a larger unit of the production of the car.
  For example the paint
  shop, which paints the car shells. These units have their own ``steward''
  who functions as a leader of the shop. The steward maintains
  discipline among
  the workers so that work is carried out according to
  the agreements between the union and the company.
  Furthermore, the steward makes sure that work is
  evenly distributed among the workers in the workshop
  and takes the other workers' interest
  higher up in the organisation hierarchy. If any of the workers have
  a problem, they go to the steward who will take the matter further
  up the organisation. This way, the production team has its own
  organisational and hierarchal structure~\cite[109-150]{b}.

  There is one important difference between this organisation and the
  overall factory organisation however. And that is how the management
  is selected. In the workshop, the steward is selected
  by the other workers.
  If they think the steward is doing a bad job, they can replace him
  with another they think is better suited for the task.
  The workshops is therefore a meritocracy as opposed to the
  factory management who are appointed following the more traditional
  employment scheme, using qualifications and seniority~\cite[129-150]{b}.

  \section{Conditions for Good Teamworking}

  Dividing workers into group is not enough in itself to achieve the
  desired outcome of the organisational change, that is increased work
  experience for the workers and increased production for the management.

  According to Locke, the worker needs a clear goal setting for
  his or her work. The more precise the goal is, the better. This means
  that it is not good enough to say ``do your best'' to get the 
  desired effect ~\cite{mcc}. He differs between two types of
  goal setting, namely challenging and specific goals. The challenging
  goals are there for the workers to have something to stribe after,
  something to which they must better themselves to accomplish and
  thereby feel that they are in development themselves, that they
  improve their own abilities. This contributes to the intrinsic
  values to the workers. Specific goals are more short term
  and describes in numbers what the worker should produce in the
  immediate future~\cite[22]{vd}.

  Locke also emphasise two other requirements for successful management
  of groups, namely participation in the goal setting and feedback on
  past performance. Participation in defining the
  goals gives the worker a higher commitment to meeting this goals.
  However, high performance can also be accomplished if
  management makes a good
  case on the reasons and objectives for the goal setting. Feedback on
  what the workers have done in the past is also valuable as it gives
  updated information on the worker's job status and is 
  motivational~\cite[252]{hb}.

  The latter point can also be found in J. Richard Hackman's theories on
  team based job design. In addition good feedback, the team need to
  be given a variety of non trivial tasks, which must be a identifiable
  piece part of the production with a visible outcome. Furthermore, the
  team must be given a certain amount of autonomy to how they are to
  function as a group and do their tasks. Lastly, Hackman lists
  that the outcome from the work group, must have a significant impact
  on other people. In order words, what the group does must have
  identifiable consequences for other workers in order to make
  their work meaningful and motivational~\cite[372]{hb}.

  This theory is supported by the interviews done by Huw Beynon at the
  Halewood Ford car plant. For many workers at Ford, the only satisfaction
  that the workers got at the car plant, was the sense of
  helping out others~\cite[119]{b}. 

  Obviously, the group members themselves are also important.
  As in all different groups, workers in a team take certain roles
  in the group setting, from which they contribute. An interesting  
  aspect of this is the composition of such a group and the
  different roles workers take. R. Meredith Belbin, cited in~\cite[329]{hb},
  identified nine different team roles, that group members take
  when entering the group setting. He claimed that in a ``perfect world'',
  all nine would be represented, even though a group necessarily
  had to consist of nine or more members. One group member can 
  play several roles and thereby giving the role a complete
  set of roles.

  \section{Conclusion}

  The change in the way organisations are structured and groups
  are developed must also be seen in conjunction with the change
  in the industry. If we look at the UK industry, which was primarily
  a manufacturing industry, has changed
  significantly the last 20 years to now being a more service
  based industry. These kinds of industry demands a different
  organisational structure and group technology found in the
  traditional model from the 1920s. Also, the principle of
  scientific management is not applicable to all kinds of work, thus
  other ways of organising workers had to be found.

  Another point, is the increased focus on employee involvement,
  job satisfaction and motivation. Employees demand more from the
  employers now, wanting both a job that can secure their
  financial situation and give them job satisfaction, supporting Maslow's 
  theory of self actualisation~\cite[241]{hb}.
  Some writers claim that we are moving
  towards a ``leisure society'', a ``post-industrial society'',
  which both implies changes in both the organisation of work and
  in the work ethics~\cite[68-70]{nb}.

  One of the key arguments for dividing the workforce into groups,
  is to get rid of the hierarchical structure of organisations and thereby
  make the organisation more dynamic and a better place to work.
  Management likes to talk about a ``flat structure''.  However,
  the same hierarchical structure is created all over again with the
  introduction of groups. It is just of a smaller scale. People
  will always take the different roles of a group, no matter its
  size. Some people will be leaders, some will be followers. Sometimes
  with consent, other times not~\cite[329]{hb}. The bottom line is
  that even though we might not like hierarchical structures, they are
  inevitable and organisations must therefore make the most out of
  making the organisation a livable environment to work within.
  What this implies, depends on the different workplaces. Nevertheless,
  it is a good thing that the focus is more on people than it has ever been.
  For companies are just about people when it comes down to it. People and
  a lot of money.

  % \tableofcontents
  \bibliography{essay}
%  \bibliographystyle{apalike}
%  \bibliographystyle{jmr}
  \bibliographystyle{plainnat}

  \nocite{*}

\end{document}